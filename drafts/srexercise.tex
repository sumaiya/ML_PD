%%%%%%%%%%%%%%%%%%%%%%%%%%%%%%%%%%%%%%%%%%%%%%%%%%%%%%%%%%%%
%
% An example LaTeX document for seniors completing their
% senior exercise in Computer Science at Pomona College.
%
% Rett Bull
% original document August 4, 2006
% modified May 30, 2007
% modified July 17, 2007
% modified August 8, 2007
% modified June 5, 2008
% minor modifications June 1, 2009
% annual modifications June 4, 2010
% annual modifications, May 10 and August 17, 2011
%
%
%%%%%%%%%%%%%%%%%%%%%%%%%%%%%%%%%%%%%%%%%%%%%%%%%%%%%%%%%%%%


%%%%%%%%%%%%%%%%%%%%%%%%%%%%%%%%%%%%%%%%%%%%%%%%%%%%%%%%%%%%
%
% The standard heading, with two commonly-used packages.
%
%%%%%%%%%%%%%%%%%%%%%%%%%%%%%%%%%%%%%%%%%%%%%%%%%%%%%%%%%%%%
\documentclass[finalcopy]{srpaper}

\usepackage{url}
\usepackage{graphicx}


%%%%%%%%%%%%%%%%%%%%%%%%%%%%%%%%%%%%%%%%%%%%%%%%%%%%%%%%%%%%
%
% The information for the front matter.
%
%%%%%%%%%%%%%%%%%%%%%%%%%%%%%%%%%%%%%%%%%%%%%%%%%%%%%%%%%%%%
\title{A Guide through the Senior Exercise}
\author{Everett L. Bull, Jr.}
\date{August 17, 2011}
\advisor{Professors Kim Bruce and Tzu-Yi Chen, advisors}
\abstract{Completion of a senior exercise in one's major is
  a graduation requirement at Pomona College. In Computer
  Science, the exercise may take one of three forms: a
  project, a thesis, or participation in a clinic.  This
  document serves as a guide to assist students in planning
  planning and executing their senior exercises. It is
  also an example of the format expected of the senior
  paper or thesis.}
\acknowledgment{The author is deeply grateful to Professors
  Bruce, Chen, Kauchak, and Sood for their patient support and
  generous suggestions, and to Patrick McNally for careful
  copy-editing.}


%%%%%%%%%%%%%%%%%%%%%%%%%%%%%%%%%%%%%%%%%%%%%%%%%%%%%%%%%%%%
%
% START IGNORING HERE!
%
% Ignore everything down to "STOP IGNORING HERE!" It is
% local customization that is unnecessary to a typical
% senior paper or thesis.
%
%%%%%%%%%%%%%%%%%%%%%%%%%%%%%%%%%%%%%%%%%%%%%%%%%%%%%%%%%%%%
%
% The following lines provide the magic to put links in
% the on-line pdf version of the document. To enable the
% magic, simply remove the comment symbol from before the
% \usepackage[pdftex]{hyperref} specification.
%
% Without the hyperref package, \plainXXX is a synonym
% for \XXX, \phantomsection does nothing, and
% \namedref{Section}{label} produces the same thing as
% Section~\ref{label}.
%
% With the hyperref package, \plainXXX produces a
% reference or url without a link, while \XXX creates
% the link. The command \phantomsection is used to
% insure that links refer to the correct page. The
% command \namedref{Section}{label} produces the same
% text as above, but the link is all of
% ``Section~\ref{label},'' giving the user more area
% ``clicking area.''
%
%\usepackage[pdftex]{hyperref}
\makeatletter
\@ifundefined{hyperref}{%
   \def\hyperref[#1]{}
   \let\plainref\ref
   \let\plainpageref\pageref
   \let\plainurl\url
   \let\phantomsection\relax
   }{%
   \newcommand{\plainref}{\ref*}
   \newcommand{\plainpageref}{\pageref*}
   \let\plainurl\nolinkurl
   \hypersetup{letterpaper=true,
               plainpages=false,
               pageanchor=true,
               breaklinks=true,
               bookmarks=true,
               bookmarkstype=toc,
               bookmarksopenlevel=2,
               bookmarksnumbered=true,
               hyperindex=true,
               colorlinks=true,
               linkcolor=blue,
               urlcolor=magenta,
               citecolor=green,
               pdftitle=\@title,
               pdfauthor=\@author}}
\makeatother
\newcommand{\namedref}[2]{\hyperref[#2]{#1~\plainref{#2}}}
\newcommand{\namedpageref}[2]{\hyperref[#2]{#1~\plainpageref{#2}}}

%
% We change the second-level ``bullet'' from a dash to a
% diamond.
%
\renewcommand\labelitemii{$\diamond$}

%
% The following definition is from Oren Patashnik's BibTeX
% documentation.
%
\def\BibTeX{{\rm B\kern-.05em{\sc i\kern-.025em b}\kern-.08em
    T\kern-.1667em\lower.7ex\hbox{E}\kern-.125emX}}

%
% These commands create the index. We change the index
% environment a bit to match our spacing and conventions.
% The multicol package does a better job of producing
% multiple columns than the standard LaTeX \twocolumn.
%
\usepackage{makeidx,multicol}
\makeindex
\newcommand{\selfreferencejoke}{% assumes the index is
                                % short enough so that
                                % ``self reference'' appears
                                % on the last page
  \index{self reference}}
\makeatletter
\renewenvironment{theindex}
  {\chapter*{\indexname}
   \addcontentsline{toc}{chapter}{\indexname}
   The index appears here as a convenience to
   those using this guide. Normally, a paper
   or research thesis will not have an index.
   \setlength{\columnseprule}{\z@}
   \setlength{\columnsep}{35\p@}
   \thispagestyle{plain}
   \begin{multicols}{2}
     \setlength{\parskip}{\z@ \@plus .3\p@}
     \setlength{\parindent}{\z@}
     \let\item\@idxitem
     \displayspacing
     \small}
  {\selfreferencejoke\end{multicols}\clearpage}
\makeatother

%%%%%%%%%%%%%%%%%%%%%%%%%%%%%%%%%%%%%%%%%%%%%%%%%%%%%%%%%%%%
%
% STOP IGNORING HERE!
%
%%%%%%%%%%%%%%%%%%%%%%%%%%%%%%%%%%%%%%%%%%%%%%%%%%%%%%%%%%%%



%%%%%%%%%%%%%%%%%%%%%%%%%%%%%%%%%%%%%%%%%%%%%%%%%%%%%%%%%%%%
%
% The document begins here.
%
%%%%%%%%%%%%%%%%%%%%%%%%%%%%%%%%%%%%%%%%%%%%%%%%%%%%%%%%%%%%
\begin{document}
\frontmatter


%%%%%%%%%%%%%%%%%%%%%%%%%%%%%%%%%%%%%%%%%%%%%%%%%%%%%%%%%%%%
%
% Preface
%
%%%%%%%%%%%%%%%%%%%%%%%%%%%%%%%%%%%%%%%%%%%%%%%%%%%%%%%%%%%%
\preface
The Computer Science Department welcomes you to the senior
exercise! We want you to have a rewarding and memorable
experience as you complete your major with a valuable,
rigorous, and fun final endeavor.

The \textit{Pomona College Catalog}~\cite{PomonaCatalog}
states, ``Each student's major will culminate in a senior
exercise designed to deepen understanding and integrate the
content and method of his or her field of study.'' Besides
being a vehicle that deepens understanding and integrates
content and method, the senior exercise ought to be a labor
of love. We encourage you to begin thinking about your
project in your junior year. Be creative. Speak to other
students and faculty members and investigate several
possibilities. This is the time to think broadly.

There are three options for the senior exercise in Computer
Science, described in detail in
\namedref{Chapter}{Chapter:Premeditation}. You may undertake
a project, develop a research thesis, or participate in a
clinic. Your first decision is to select an option. Then, in
the fall semester of your senior year, you will enroll in
the Senior Seminar and begin work on your senior
exercise. Depending on the option you select, you may enroll
in another course besides the seminar. Your senior exercise
will culminate in the spring of your senior year with a
presentation and a written document; see
\namedref{Chapters}{Chapter:Presentation}
and~\ref{Chapter:Prose}.

Early in the process, you will be paired with an advisor who
will be a Computer Science faculty member at Pomona
College. Working closely with your advisor, you will
determine the shape of your senior exercise.

In this guide, we have concentrated on the \emph{decisions}
you must make and the \emph{mechanics} of the process, and
we have tried to avoid being overly prescriptive about the
\emph{nature} of an individual undertaking.  We hope that
this guide will be helpful as you plan, execute, and
complete your senior exercise. With thoughtful (and early)
planning, your senior exercise will proceed smoothly. Please
read this guide carefully and refer to it often; its pages
contain experience and advice collected by students and
faculty members over the past years. If you do find yourself
in a crisis, refer to the suggestions in
\namedref{Appendix}{Appendix:CrisisManagement}.


%%%%%%%%%%%%%%%%%%%%%%%%%%%%%%%%%%%%%%%%%%%%%%%%%%%%%%%%%%%%
%
% Chapter: Premeditation
%
%%%%%%%%%%%%%%%%%%%%%%%%%%%%%%%%%%%%%%%%%%%%%%%%%%%%%%%%%%%%
\chapter{Premeditation}
\label{Chapter:Premeditation}
There are three options for the senior exercise: a project,
a research thesis, and a clinic experience. The project has
been available for several years, while the thesis and
clinic options were first offered in 2007--2008.

Most of this chapter is devoted to detailed outlines of the
three options, including the requirements, the schedule, the
advisor, and the grading policies. There are some
redundancies among the descriptions because we want each one
to be self-contained. You can use the outline as a framework
and a checklist as you work through the senior
exercise. Read these outlines in conjunction with the list
of approximate dates in \namedref{Appendix}{Appendix:Schedule}.
The specific dates for the current academic year will be
announced in the Senior Seminar, over e-mail, and on the
Department's web site.

 
\section{Common Features}
Each senior must take Computer Science~190,
\textit{Senior Seminar,}\index{senior seminar}
which is a separate graduation
requirement. The senior exercise component is a part
of the seminar, and the grade in the senior seminar
is based in part on preparations for the senior exercise.
In the research thesis and clinic options,
one of the required courses may count as an elective toward
the major requirements.

Each senior must attend the departmental
colloquia\index{colloquium}, which
are held approximately every other week on Thursday at
4:15~pm. If for any reason you have to miss a colloquium,
you may make it up by attending a colloquium at Harvey Mudd
College or another event. Make the arrangements with the
faculty member who is overseeing the senior exercise.

Additionally, each senior must participate in monthly
progress meetings\index{progress meetings},
give a presentation toward the end of the spring semester of
their senior year, and will submit some form of written
work.

Finally, each senior will have an advisor\index{advisor}
who is a member of
the Pomona College Computer Science faculty. The advisor for
the senior exercise may or may not be the same as the
student's academic advisor. The role of the advisor will
vary according to the option, but in all cases, the advisor
will guide the student through the ``local requirements,''
including the progress meetings, the presentation, and the
written work.

Although the work for the senior exercise is done in several
courses, one course in each option is identified as the
``senior exercise course'' to satisfy the college
requirement that a senior exercise receive at least one-half
course credit.


\section{Project}
\index{project|(}
A project is currently the default method of satisfying the
senior exercise requirement. The project is carried out
(normally) in the spring semester of the senior year, with
some of the preparation being done in the preceding
semester. A project can consist of software development,
library research, or theoretical study. In all cases, there
will be significant background research.

The role of a project advisor\index{advisor!role of}
varies according to the
student, the advisor, and the topic. The advisor is
sometimes quite directive, often a source of
valuable advice, and frequently a sounding
board for the student's ideas. In any case, you should
expect to meet with your advisor regularly, at least weekly,
over the course of the project. You and your advisor should
develop a detailed schedule to keep the
project on track.

The advisor\index{advisor|(}
must be a faculty member in the Computer Science
Department of Pomona College. You may, of course, seek
advice from faculty members in other departments or at other
colleges. In some cases, students have worked most closely
with someone other than their official departmental advisor.

Matching a student with an advisor and a topic is a
complicated process that depends, in part, on the student's
interest. The student should, normally, have taken one or
more advanced courses in the general area of the topic, and
the advisor should have some expertise in that area. The
student and the advisor should be comfortable working with
one another. Also, because we want to divide the workload
evenly among advisors, each potential advisor has a limit on
the number of advisees. In mid-September students are
asked to submit a ranked list of three or more 
advisor-project pairings, and with understanding and good
will, we expect to come to a fair
and beneficial assignment. In any case, each student will
have an advisor\index{advisor|)} 
and a topic by about the third week of the
fall semester of the senior year.

\subsection*{Requirements for the Project}
\index{project!requirements}%
The requirements for the project option are as follows:
\begin{itemize}
\item
Computer Science~190, \textit{Senior Seminar,} in the fall
semester. In addition to the other work of the seminar, each
student will
\begin{itemize}
\item
submit a list of advisor-topic\index{advisor!choice of} pairs
in the first weeks of the semester;
\item
complete the background research for the project and
submit an annotated bibliography and a literature review
at some point
in the second half of the semester;
\item
acquire and install any software, data, or
tools that are needed for the project; and 
\item
submit an extended abstract with a specific description
of the proposed project's purpose and methods.
\end{itemize}
\item 
Participation in the (approximately monthly) progress
report meetings, both semesters. In the fall semester, the
meetings will be part of the Senior Seminar.
\item
Attendance at the (approximately biweekly) departmental
colloquia, both semesters.
\item
Computer Science~192, \textit{Senior Project,} in the spring
semester. Most of the work on the project takes place in the
first two-thirds of the semester. Each student will
\begin{itemize}
\item
submit, shortly after the middle of the semester, a
substantial draft that includes all the sections of the
paper;
\item
give a formal presentation, near the end of the spring
semester, of the project, experience, and conclusions; and
\item
submit a paper which reports on the project and its results.
\end{itemize}
\end{itemize}
This document itself serves as a model for the form of the
project report; see \namedref{Chapter}{Chapter:Prose}.

\subsection*{Evaluation of the Project}
\label{Section:ProjectEvaluation}
\index{project!evaluation}
Computer Science~190 and~192 count as one-half course
each. Computer Science~192 is the designated senior exercise
course.

% This language is adapted from Kim Bruce's email of Feb 5, 2008.
A student's grade in Computer Science~192 is determined jointly
by the Computer Science faculty members who consider six
components of the project:
\begin{itemize}
\item The goal of the undertaking. Was it appropriately
ambitious?
\item Preparation. Did the literature review and other 
preparatory activities cover the appropriate prior work in
the area?
\item Execution. Were the efforts (both in quantity and
quality) sufficient? Were all the intermediate and final
deadlines met?
\item Evaluation of results. Was the process of evaluation
appropriate and thorough?
\item Presentations. Were the oral and written presentations
clear, understandable, and complete? Keep in mind that
presentations are graded on \emph{both} content and delivery.
\item Participation. Did the attendance and contribution to
group meetings, conferences with the advisor, and the final
presentation meet faculty expectations?
\end{itemize}
These items are interrelated, and there is no set number of 
``points'' assigned to each one. In particular,
there is an interaction between the first and third items.
A wonderful execution of a trivial goal may not be worth 
more than a weak outcome on a more challenging goal.  And a
ridiculously challenging goal may preclude you from any success
at all. The key is to find the right balance.

Further, you should keep in mind that a successful project need
not be one that is clearly better than all previous work in the
area.  We often learn as much or more from experiments that fail
as from those that succeed.  On the other hand, we do expect you
to make your project as successful as possible within the time
constraints.

\phantomsection
\index{grading}
To state the criteria in another way,
an\label{Page:ProjectGrading} \textbf{A-range} project is
interesting, well-conceived, ambitious, and
successful. There is a significant result, although it need
not exactly match the original goal of the project. Any
obstacles that arise are handled thoughtfully and
creatively. All work is submitted on time. The presentation
and the paper are clear, polished, and
well-organized. In order to receive an A, all the
components---conception, preparation, execution,
presentation, and written work---must be of A~quality.

A \textbf{B-range} project is fairly ambitious but only
partly successful, or else it is less ambitious and solidly
successful. There are results, but they may be incomplete or
tentative. All the requirements and deadlines are met. The
presentation and the paper or thesis are clear and
organized.

A \textbf{C-range} project is insubstantial or falls short
of its goals; there are few definite results. All the
requirements are met, even if some of the work is a little
late. The presentation and the paper or thesis are
disorganized or occasionally muddled.

A \textbf{failing} project is poorly conceived and
inadequately executed. It is lacking substance that might
lead to a solid and interesting conclusion. Some of the work
may be seriously late or missing altogether. The
presentation and paper are ungrammatical or difficult to
follow.

\phantomsection
In\label{Page:SeniorExerciseGrading} addition to the grades
in the various courses, the Pomona College Senior Exercise
is graded ``not passed,'' ``passed,'' or ``passed with
distinction.'' The designation appears on the transcript. It
is rare, but not without precedent, for a student to fail
the senior exercise and therefore not graduate. At the other
end of the spectrum, ``distinction'' is reserved for a small
number of really outstanding senior exercises. A grade of A
in the senior exercise course does not guarantee a rating of
``distinction'' on the senior exercise.\index{project|)}


\section{Group Project}
\index{group project|(}
In 2010-2011, the Computer Science department will offer the option 
of a group project with Professor Bruce as the advisor. A team of up
to five students will collaborate on developing software for mobile
devices. If the experiment is successful, the group project may become a
standard option for the senior exercise in future years.

The procedural details for the group project are similar to those for
the project and will be worked out in more detail during the year.
In place of background research and the annotated bibliography,
students will learn about the languages and systems to be used.
In the case of mobile devices, network communications and human
interface design are important issues to be studied. The extended
abstract will contain a complete design of the software and
allocation of tasks to individual students.

Students who are interested in the group project will list it
as one of the advisor-project pairs that is submitted in
mid-September. The team will have at most five members, and it
may have fewer. There is no guarantee that
everyone who wants to participate in the group will be
selected.

\subsection*{Requirements for the Group Project}
\index{group project!requirements}%
The requirements for the group project option are as follows:
\begin{itemize}
\item
Computer Science~190, \textit{Senior Seminar,} in the fall
semester. In addition to the other work of the seminar, the
members of the group will
\begin{itemize}
\item
learn the fundamentals and design principles that will guide
the project,
\item
become competent with the software tools to be
used,
\item
create a detailed design of the system to be created, and
\item
assign specific spring-semester tasks to team members.
\end{itemize}
\item
Attendance at regular meetings with the group and its advisor. 
The team will meet weekly in the fall semester and more
frequently in the spring. The advisor will attend at least
one meeting a week.
\item
Participation in the (approximately monthly) progress
report meetings, both semesters. In the fall semester, the
meetings will be part of the Senior Seminar.
\item
Attendance at the (approximately biweekly) departmental
colloquia, both semesters.
\item
Computer Science~192, \textit{Senior Project,} in the spring
semester. Most of the work on the group project will occur in the
first two-thirds of the semester. As with the regular project,
there will be a paper that reports on the project and its results
and a formal presentation. The extent and content of the paper
and presentation will be determined in consultation with the
advisor.
\end{itemize}

\subsection*{Evaluation of the Group Project}
\index{group project!evaluation}
As in the regular project, Computer Science~190 and~192 count as one-half course
each, and Computer Science~192 is the designated senior exercise
course. A student's grade in Computer Science~192 is determined jointly
by the Computer Science faculty members based on criteria similar to
those for the regular project; see \namedref{Section}{Section:ProjectEvaluation}.
The grade will be based both on the individual team member's performance \emph{and}
the overall effectiveness of the group.
\index{group project|)}


\section{Research Thesis}
\index{research thesis|(}
The research thesis option is for those students who want to
investigate a topic more deeply. The topic is an aspect of
the thesis advisor's research program, and the work is
carried out in close cooperation with the 
advisor.\index{advisor|(} The
finished product will be a substantial written thesis. A
research thesis is a substantial undertaking that involves
a full course in each semester of the senior year.

Students wishing to satisfy the senior exercise through the
research thesis option must obtain permission from the
department early in the spring semester of their junior
year. Students who are interested in the thesis should
approach a potential advisor in plenty of time to discuss
the possibilities. Often preparatory research is done in the
junior year or during the summer before the senior year.

The research thesis option is excellent preparation for a
student who is considering graduate school. It is intended
for students who are passionate about a particular topic,
can work independently, and are well-equipped to pursue the
project. Because of the intense and personal nature of the
research thesis, the student and thesis 
advisor\index{advisor|)} must
mutually agree to work together. We cannot guarantee that
every student will have the opportunity to do a research
thesis.

\subsection*{Requirements for the Research Thesis}
\index{research thesis!requirements}
\begin{itemize}
\item
Computer Science~190, \textit{Senior Seminar,} in the fall
semester. In addition to the other work of the seminar,
each student will complete the background research for the
thesis and submit an annotated bibliography, a
literature review, and an extended abstract in the second
half of the semester.
\item 
Participation in the (approximately monthly) progress report
meetings, both semesters. In the fall semester, the meetings
will be part of the Senior Seminar.
\item
Attendance at the (approximately biweekly) departmental
colloquia, both semesters.
\item 
Computer Science~191, \textit{Senior Research and Thesis in
Computer Science,} in the fall semester.
\item
Computer Science~191, \textit{Senior Research and Thesis in
Computer Science,} in the spring semester. Each student will
\begin{itemize}
\item
submit, shortly after the middle of the semester (or earlier
if the advisor requests it), a substantial draft that
includes all the sections of the thesis;
\item
give a formal presentation, near the end of the spring
semester, of the thesis, experience, and conclusions; and
\item
submit the completed thesis at the end of the semester.
\end{itemize}
\end{itemize}
This document itself serves as a model for the form of the
thesis; see \namedref{Chapter}{Chapter:Prose}.

\subsection*{Evaluation of the Research Thesis}
\index{research thesis!evaluation}
Computer Science~190 counts as one-half course. Computer
Science~191 counts as a full course each semester. The fall
semester of that course may count as an elective toward the
major requirements; the spring semester is the designated
senior exercise course.

The grading standards for Computer Science~191 are the same
as for the project; see the descriptions on
\namedpageref{page}{Page:ProjectGrading}. Obviously, the
research thesis is expected to be longer and more
substantial than the paper for a project. The thesis
advisor\index{advisor}
will have a significant role in the determination of the
grades for Computer Science~191.

As stated on
\namedpageref{page}{Page:SeniorExerciseGrading}, students
will also receive a grade for the overall senior
exercise---not passed, passed, or passed with distinction.%
\index{research thesis|)}


\section{Clinic}
\index{clinic|(}
The Harvey Mudd College Computer Science Clinic is a
program in which a corporation or research organization
sponsors an academic-year-long effort to solve a real-world
problem. A team of three to five students, together with a
faculty advisor and a liaison from the client, works on the
problem and provides written and oral reports to the
client. For more information, see
\url{http://www.cs.hmc.edu/clinic}.

A Pomona College student may satisfy the senior exercise
requirement by participating in a clinic project. The clinic
experience requires participation in both semesters of a
Computer Science Clinic project at Harvey Mudd
College. Students taking this option will work closely with
the project advisor and the other team members. Each student
will have, in addition, a Pomona College 
advisor\index{advisor} to provide
guidance through the local requirements.

The clinic option is well-suited for students who want to
participate in a team effort or students who are interested
in ``trying out'' a career in industry.

Computer Science~121, \textit{Software Development,} is a
prerequisite for the clinic.\index{clinic!prerequisite}

Students wishing to satisfy the senior exercise through the
clinic option must obtain permission from the department
early in the spring semester of their junior year. An early
commitment is necessary because Harvey Mudd College plans
its clinic projects well in advance. Students must
preregister in the spring to take Computer Science~183 in
the fall semester of their senior year.

\subsection*{Requirements for the Clinic}
\index{clinic!requirements}
\begin{itemize}
\item
Computer Science~190, \textit{Senior Seminar,} in the fall
semester.
\item
Participation in the (approximately monthly) progress report
meetings, both semesters. In the fall semester, the meetings
will be part of the Senior Seminar.
\item
Attendance at the (approximately biweekly) departmental
colloquia, both semesters.
\item
Computer Science~183 and~184, \textit{Computer Science
Clinic I and II,} in the fall and spring semesters,
respectively. A student must participate in all aspects of
the clinic, including the reports and presentations to
sponsors. The clinic is a full-year commitment; a student
who enrolls in the fall is expected to participate through
the completion of the project in the spring.
\item
An individual presentation, near the end of the spring
semester, describing the clinic project, its successes and
failures, and its results.
\item
A copy of the final clinic report. The clinic report
will serve as the written record of the senior exercise; a
separate individual report is not required.
\end{itemize}

\subsection*{Evaluation of the Clinic}
\index{clinic!evaluation}
Computer Science~190 counts as one-half course. Computer
Science~183 and~184 each count as a full course.
Computer Science~183 may count as an elective
toward the major requirements; Computer Science~184 is the
designated senior exercise course.

Students following the clinic option are, of course, graded
by their clinic advisor in Computer Science~183 and 184 in
accordance with the policies of the Harvey Mudd College
Computer Science Clinic.

As stated on \namedpageref{page}{Page:SeniorExerciseGrading},
students will also receive a grade for the overall senior
exercise---not passed, passed, or passed with
distinction. The grade will be determined by the Pomona
College Computer Science faculty in consultation with the
advisor of the clinic project.\index{clinic|)}


%%%%%%%%%%%%%%%%%%%%%%%%%%%%%%%%%%%%%%%%%%%%%%%%%%%%%%%%%%%%
%
% Chapter: Preparation
%
%%%%%%%%%%%%%%%%%%%%%%%%%%%%%%%%%%%%%%%%%%%%%%%%%%%%%%%%%%%%
\chapter{Preparation}
\label{Chapter:Preparation}
This chapter is written \emph{primarily} for students
following the project and research thesis options. By the
beginning of the fall semester of the senior year, everyone
will have an advisor\index{advisor} and a
topic.\index{topic!choosing} The research thesis
students will have made those decisions much earlier. As
part of the work for the Senior Seminar, project and
research thesis students will carry out background research
and submit annotated bibliographies and literature reviews,
as described in
\namedref{Sections}{Section:AnnotatedBibliography}
and~\ref{Section:LiteratureReview}.

Those participating in a clinic have already made
decisions---one way or another---about advisors and topic,
and the course of their work follows a different
schedule. Everyone, though, can benefit from the comments on
scheduling in \namedref{Section}{Section:Scheduling}.


\section{Advisor and Topic}
\label{Section:AdvisorAndTopic}
\index{topic!choosing}
\index{advisor|(}
The first task is to find an advisor and a topic. Sometimes,
students have a clearly-formulated topic in mind and look
for an advisor who will help them refine and develop it. In
other cases, students select an advisor first and then
investigate various topics. There is nothing wrong with
``shopping around'' at the beginning of the process. Be
willing to speak with several possible advisors and hear
their suggestions for topics.

The choice of a topic requires careful thought. Keep an open
mind as you search.  Although many students write long
programs, it is not necessary to write any code at all. A
project or thesis might be a collection of data on how
humans use computers, a study of legal or social issues
connected with information technology, a statistical
analysis of network performance, a proof of correctness or
careful computation of complexity for an algorithm, a design
of an aspect of a programming language, or an exposition of
one of the deep ideas in Computer Science.

Look for a topic that is interesting, deep, specific, and
feasible. The topic should, first and foremost, be
\emph{interesting} to you. After all, you will be spending a
whole semester (or a whole academic year, in the case of a
research thesis) working on it. In the spirit of the Pomona
College Senior Exercise, the undertaking should ``deepen
your understanding'' and ``integrate the methods and
content'' of Computer Science.

The topic should also be \emph{deep} in the sense that it
builds on the courses you have taken for your Computer
Science major. A student who just finished a couple of
introductory courses would not be prepared to complete a
project or thesis on such a topic. The most successful
undertakings are ones in which the student has completed one
or more advanced courses in the area. It is also desirable
to choose a topic close to your advisor's areas of
expertise.

Generally, a more modest undertaking that is well-done is
preferable to an ambitious but haphazardly-executed one. On
the other hand, an utterly trivial task will not receive an
A, no matter how well it is done.

The project or thesis proposal must be \emph{specific} in
that it attempts to answer a question, illustrate a
principle, explain an important idea, or conduct an
experiment. It is not enough just ``to write a program'' or
``to try something and see what happens.'' When it comes
time to present your work, you want to have a clear and
specific conclusion.

As you evaluate possible topics, think about what kind of
conclusion is appropriate. Is it a statistical analysis, a
proof, a graphical display, or something else? Whatever it
is, it will take some effort to create.

Finally, the project or thesis must be \emph{feasible.} On
one hand, you will be working on it for \emph{a whole
semester} (or \emph{a whole year,} in the case of a
thesis). On the other, you have \emph{only one semester} to
do it, and you will have other courses and activities during
that time.  And as you get to the end, you will have
\emph{only a few weeks} to finish it.  Many projects and
theses, but not all of them, involve intensive coding. If
you are thinking of writing code, keep in mind that
(1)~coding takes longer than you think it will, and
(2)~there are other parts of the project that are equally
important. Even if you do not plan to write programs, make
realistic estimates of the time required to complete the
project.

We all understand that your ideas and your work will
evolve. Stay flexible and expect to change your direction
slightly as you learn more about the subject. Your plans may
even include decisions to be made on the basis of early
work. Careful planning at the beginning will, however,
reduce the possibility that you will have to throw out
everything and start over.

Your advisor\index{advisor|)} can help you to achieve the
right equilibrium among the competing qualities of
interest, depth, precision, and feasibility.


\section{Annotated Bibliography}
\label{Section:AnnotatedBibliography}
\index{annotated bibliography}
Once you have decided on a topic, your first task is to
begin collecting information about it. (That process may, in
fact, begin earlier---as you are casting about for a
subject.) It is highly unlikely that you are the first
person to be interested in the subject. Find out what others
have done. Who was the first to raise the general question?
What are the significant developments in the subject? Have
others tried to solve similar problems? If so, in what ways
were they successful? What is the origin of your specific
approach, and what are the competing approaches? How does
your project differ from previous work?

Work with your advisor to get started. Often, once you find
a few current articles, you can use their reference lists to
work your way back in time to the original works on the
subject.

For each article you find, keep a record of the full
bibliographic citations and a summary of the
article---anywhere from a short phrase to a few
paragraphs. \namedref{Figure}{Figure:SampleAnnotatedBibliography}
shows three annotated records: a journal article, a piece of
software, and a submission to a conference.
\begin{figure}
\begin{center}
\begin{minipage}{0.9\textwidth}
% We manually modify the bibliography environment.
\renewcommand{\chapter}[2]{}
\renewenvironment{quotation}{\par}{}
\footnotesize
\begin{thebibliography}{GJS76}
\bibitem[GJS76]{GibbsPoSt76}
N.~E. Gibbs, W.~G.~Poole Jr., and P.~K. Stockmeyer.
\newblock An algorithm for reducing the bandwidth and
  profile of a sparse matrix.
\newblock {\em {SIAM} J. Num. Anal.}, 13(2):236--250, April
  1976.
\begin{quotation}\noindent
  Looks at improving the RCM algorithm [LiuSh76].  Basically
  original cm and rcm algorithms suggest trying \emph{many}
  start nodes and taking the one which leads to ordering of
  smallest bandwidth. They show, in part, how to reduce the
  number of bfs's that are done.\par
  Pseudo-diameter: run the bfs once, then take the nodes in
  the last level and run bfs again using each of those last
  nodes as the new root. Goal is to end up with two
  endpoints of a pseudo-diameter. They provide an algorithm
  for merging bfs trees from each endpoint (helps minimize
  width).\par They demonstrate the algorithm on an
  assortment of structured grids.
\end{quotation}
\par\vspace{0.5\topsep}
\bibitem[IPM]{IPM05}
{IPM}: integrated performance monitoring.
\newblock \plainurl{http://www.nersc.gov/projects/ipm}.
\begin{quotation}\noindent
  Describes a tool that generates a detailed performance
  profile for parallel codes. Low overhead, no source code
  modification, summarizes computation and communication
  costs. [checked web site 6/1/07]
\end{quotation}
\par\vspace{0.5\topsep}
\bibitem[SSl05]{SchwartzStWe05}
J.~Schwartz, A.~Steger, and A.~Wei\ss l.
\newblock Fast algorithms for weighted bipartite matching.
\newblock In S.~E. Nikoletseas, editor, {\em Proceedings of
  the 2005 Workshop on Experimental Algorithmics (WEA
  2005)}, volume 3503 of {\em LNCS}, pages 476--487, 2005.
\begin{quotation}\noindent
  Asked by a company to come up with a faster way to compute
  minimum weight perfect matchings on complete bipartite
  graphs ($n>10000$). Standard algorithms ($O(n^3)$ for real
  weights, $O(\sqrt{n}m\log(nC))$ for integer weights) too
  expensive.\par They experiment with discarding a large
  subset of the edges in the following way: first keep the
  smallest $(c\log n)$ smallest edges adjacent to each
  vertex, then add in $c' n\log n$ uniformly at random
  chosen edges of $E$. The second means that, by a theorem
  of Erd\"os and R\'enyi, the new graph has a perfect matching
  with high probability.\par Then can use sparse algorithms
  to compute perfect matching. For arbitrary edge weights
  can get upper bound on achieved approximation factor based
  on choice of $c$. For uniformly distributed edge weights
  can show will get optimal matching with high probability
  based on a theorem by Frieze and Sorkin which bounds the
  heaviest edge in a min-weight perfect matching for a
  complete bipartite graph in this case.\par In tests on
  real data, find if highly structured, need $c>120$,
  otherwise 15 ok. $c'$ was set small ($=4$). If generate
  random uniformly distributed edge weights, cutoff value of
  $c=4$ was enough.
\end{quotation}
\end{thebibliography}
\end{minipage}
\end{center}
\caption[Sample annotated bibliography]{Three entries from
an actual annotated bibliography, courtesy of Professor
Tzu-Yi Chen.}
\label{Figure:SampleAnnotatedBibliography}
\end{figure}

Evaluate each article critically. Is there a well-defined
problem or question? What is the article's conceptual or
theoretical framework? Is the approach or methodology
appropriate and effective? Are the experiments repeatable?
(You might even try repeating one or two.) Is the
presentation clear and logical? Do the data or the arguments
support the conclusions? What contribution does the article
make to your topic?

The annotated bibliography will, ultimately, contain all the
works connected to your topic. You may omit those papers you
found to be irrelevant. By the time you have completed the
bibliography, you should have a sense that you really
understand the development of the field and the current
state of the art.

\subsection{Quality of Sources}
\index{quality of sources|(}
\index{annotated bibliography!sources|see{quality of sources}}
As you carry out your research, look for authoritative,
reliable sources. They are the works that are released by
reputable publishers, are written by authors with respected
credentials, or have gone through rigorous peer-review
processes.

The following types of works are examples of unassailable
sources:
\begin{itemize}
\item scholarly books;
\item articles in well-known, refereed journals; and
\item contributions to well-regarded, refereed conferences.
\end{itemize}
Your advisor\index{advisor!role of}
can help you understand which journals are
``well-known'' and which conferences are ``well-regarded.''
Conference papers are frequently followed
by journal articles; in those cases, you should cite the
journal article.

As a general rule, avoid citing a source that only appears
on the web \cite{SeniorExerciseGuide}.  The reference
\cite{AKS2002} is an example of such a paper. It is, by far,
preferable to cite the later publication \cite{AKS2004}. If
a published work appears also on the web, it is helpful to
the reader to give the url as well, as was done in
\cite{AKS2004}.

A valuable resource for your work is the CiteSeer database
of articles in computer science. See
\url{http://citeseer.ist.psu.edu} or the mirrors at
\url{http://citeseer.csail.mit.edu} and
\url{http://citeseer.ittc.ku.edu}. The Association for
Computing Machinery's Digital Library, at
\url{http://portal.acm.org/dl.cfm}, is available on campus
(through financial support from
Honnold Library), as are many of the publications of the
Institute of Electrical and Electronics Engineers; see
\url{http://www.ieee.org/web/publications/home/index.html}.

Some works do not appear on the web at all. This is
particularly true for some of the seminal, ground-breaking
papers from decades ago. As it is important to read and to
cite these works, you will probably have to use some of
Honnold Library's databases (just plain Google will not do
the job!), and you may even have to set foot inside a
library.

An exception to the ``no web'' rule is a work that is brand
new and has not yet been published. For example,
\cite{AKS2004} would not have been published in 2002, and
\emph{at that time} a citation of \cite{AKS2002} would have
been acceptable.

Another possible exception to the ``no web'' rule is
software. You may use software that you acquired from the
web, and you should cite the source. The item labeled [IPM]
in \namedref{Figure}{Figure:SampleAnnotatedBibliography} is
an example of such a citation. There is often a published
paper that accompanies the software (and is listed on the
site with the software), and when it exists, you should
include a reference to the paper is well.

The following works are potentially acceptable, depending
upon the situation:
\begin{itemize}
\item technical reports from major universities,
\item white papers from corporations,
\item articles from newspapers and the popular press, and
\item blogs or on-line essays from well-known authorities.
\end{itemize}
Technical reports are often preliminary versions of conference
papers or journal articles. Cite the later, refereed versions
when they are available.

Sometimes industry white papers are the only sources of 
authoritative information. Examples include the Java 
specifications from Sun Microsystems and the detailed
Pentium specifications from Intel.

News articles and blogs can be especially helpful in
providing a contextual setting or give a sense of the field
at a particular moment. However, they are unlikely
to be central to the main topic.

The following types of works are \emph{never} authoritative:
\begin{itemize}
\item writings that appear only on the web and offer no
evidence of reliability, and 
\item Wikipedia articles.
\end{itemize}
Remember that a web search is as likely to turn up the term
paper of a high school student or the blog of a wacko as it
is to find an article of a reputable expert from a known
publisher. The reason that Wikipedia articles are never
deemed reliable is that they are written anonymously by
authors whose credentials are unknown; there is no reason to
be confident that a Wikipedia article is complete or
correct. Wikipedia may be useful, however, in leading you to
more authoritative sources.\index{quality of sources|)}

\subsection{Creating the Bibliography}
\index{annotated bibliography!formatting|(}
Use \LaTeX\ and \BibTeX\ to prepare your annotated
bibliography. General information about \LaTeX\ appears in
\namedref{Sections}{Section:LaTeXIntro}
and~\ref{Section:LaTeXAnatomy}, and the \BibTeX\ format
is described in \namedref{Section}{Subsection:Bibliographies}.
The book \cite{Lamport1994} contains a comprehensive
introduction to \BibTeX.
The \BibTeX\ source for the annotated bibliography of
\namedref{Figure}{Figure:SampleAnnotatedBibliography} appears in
\namedref{Figure}{Figure:SampleAnnotatedEntries}.
\begin{figure}
\begin{center}
\begin{minipage}{0.9\textwidth}\fontsize{9}{10}
\begin{verbatim}
@Article{GibbsPoSt76,
  author =  {N.~E.~Gibbs and W.~G.~Poole~Jr. and P.~K.~Stockmeyer},
  title =   {An algorithm for reducing the bandwidth and profile
             of a sparse matrix},
  journal = {{SIAM} J. Num. Anal.},
  year = 1976,
  volume = {13},
  number = {2},
  pages = {236-250},
  month = {April},
  annotate = "Looks at improving the RCM algorithm [LiuSh76].
    Basically original cm and rcm algorithms suggest trying
    \emph{many} start nodes and taking the one which leads to
    ordering of smallest bandwidth.  They show, in part, how
    to reduce the number of bfs's that are done.\par

    ..."
}

@Misc{IPM05,
  title = {{IPM}: integrated performance monitoring},
  howpublished = {\url{http://www.nersc.gov/projects/ipm}},
  annotate = {
    Describes a tool that generates a detailed performance
    profile for parallel codes.  Low overhead, no source code 
    modification, summarizes computation and communication 
    costs.  [checked web site 6/1/07]}
}

@InProceedings{SchwartzStWe05,
  author = {J.~Schwartz and A.~Steger and A.~Wei\ss{}l},
  title = {Fast algorithms for weighted bipartite matching},
  booktitle = {Proceedings of the 2005 Workshop on Experimental
               Algorithmics (WEA 2005)},
  year = {2005},
  pages = {476--487},
  editor = {S.~E.~Nikoletseas},
  volume = {3503},
  series = {LNCS},
  annotate = "Asked by a company to come up with a faster way
    to compute minimum weight perfect matchings on complete
    bipartite graphs ($n>10000$).  Standard algorithms
    ($O(n^3)$ for real weights, $O(\sqrt{n}m\log(nC))$ for
    integer weights) too expensive.\par

    ..."
}
\end{verbatim}
\end{minipage}
\end{center}
\caption[Sample bibliographic entries]{The sources for the
bibliographic items in
\namedref{Figure}{Figure:SampleAnnotatedBibliography}. The
annotations have been truncated to save space.}
\label{Figure:SampleAnnotatedEntries}
\end{figure}

This may be your first experience with \LaTeX. The biggest
part of the task will be creating the bibliography file. 
The sample file
\texttt{srpaper\discretionary{-}{}{-}sample-biblio.bib}%
\index{sample files!annotated bibliography}%
\index{annotated bibliography!sample files}
provides an example of such a file, with many different kinds
of entries.

Once you have the bibliography file, creating the final
document is easy. Copy and rename the template file
\texttt{annbib-template.tex}%
\index{sample files!bibliography template}
and then edit it to fill in a
few fields. The comments will direct you to the relevant
places in the template file. If you renamed the sample file
to \texttt{mybib.tex}, then the command-line arguments to
produce the document appear below; see
\namedref{Section}{Section:LaTeXAnatomy}. The result is the
file \texttt{mybib.pdf}.
\begin{code}
latex mybib
bibtex mybib
latex mybib
latex mybib
\end{code}
Please send copies of the completed bibliography to your
advisor and to the instructor in charge of the Senior
Seminar. See \namedref{Appendix}{Appendix:Schedule} for
an approximate 
due date.\index{annotated bibliography!formatting|)}


\section{Literature Review}
\label{Section:LiteratureReview}
\index{literature review|(}
At the end of the fall semester, you will submit a
literature review. It will normally consist of two to six
pages of prose and a complete bibliography. If your
annotated bibliography is already a complete compilation of
the relevant sources, then all that remains to do is to
write a clear description of the history of the
subject---with citations.  In many cases, you will identify
one or two primary resources that you will consult in depth
during the spring semester. The completed review will give
you a clear sense of the context into which your work fits.

The literature review is a \emph{draft} of the
``background'' chapter of your project paper. As you
complete the project in the spring semester, you will no
doubt run across other sources and include them in your
summary of previous work.

The ability to carry out background research is
important. When you have completed your bibliography and
literature review, you will have developed valuable skills
in finding resources and evaluating them efficiently and
effectively.

Prepare the literature review using \LaTeX\ and the
\texttt{srpaper} document class with the \texttt{short}
option.\index{literature review!formatting}
See \namedref{Sections}{Section:LaTeXIntro},
\ref{Section:LaTeXAnatomy}, and~\ref{Section:LaTeXDetails}
for instructions.

The literature review is due a few weeks before the semester
ends. An approximate date appears in
\namedref{Appendix}{Appendix:Schedule}; the exact date will
be announced and posted. There will be cases 
in which the Computer Science faculty requests clarification
or elaboration of the literature review; there is a second
date in \namedref{Appendix}{Appendix:Schedule} for the
revision. As with the annotated bibliography, send copies
of the literature review to
your advisor and the instructor in charge of the Senior
Seminar.\index{literature review|)}


\section{Extended Abstract}
\label{Section:ExtendedAbstract}
\index{extended abstract|(}
Along with the literature review, you will submit an extended
abstract for your project. By the time 
you have completed the background reading and thinking, 
you should have a very concrete idea of your project.

% Some of the language below is taken from Kim Bruce's email
% to the CS majors on November 7, 2007.
The extended abstract might also be called a ``research
proposal.'' It is the sort of document that normally must be
approved by supervisors (either in academia or industry) before
a research project is undertaken, although in our case it is
significantly shorter.  It will be a separate document,
about two or three paragraphs long, which describes the
specific goals and strategies for the project. It should
tell the reader
\begin{itemize}
\item what question you intend to answer,
\item what you plan to do over the course of the project, and
\item how you plan to draw conclusions from your work.
\end{itemize}
You may plan to write code, conduct
an experiment, prove a theorem, design a system or a
language, or study the development of an idea. Whatever you
plan to do, be sure to include a strategy (including,
usually, the collection of data) that will lead to a 
solid conclusion.
Be sure to describe techniques for statistical analysis of
your data if it is appropriate for your subject, as it will
be, for example, if you are measuring the performance of a
program or computer system.

Once approved, you should think of this as a contract
between you and your advisor.  Although it may be revised
over time, it will guide the evaluation of your project.
Even though it is only a page long, you should think
carefully about what goes into it and discuss it thoroughly
with your advisor.

Like the literature review, the extended abstract is to
be prepared with the \texttt{srpaper} document class
using the \texttt{short} option.%
\index{extended abstract!formatting}
See \namedref{Sections}{Section:LaTeXIntro},
\ref{Section:LaTeXAnatomy}, and~\ref{Section:LaTeXDetails}
for instructions.

The extended abstract is due at the same time as the 
literature review. The approximate due dates for the
abstract and, if necessary, the revision appear in
\namedref{Appendix}{Appendix:Schedule}. Send copies of
the extended abstract to your advisor and the instructor
in charge of the Senior Seminar.%
\index{extended abstract|)}


\section{Scheduling}
\label{Section:Scheduling}
\index{scheduling|(}
Naturally, the largest part of your effort will go into the
execution of the project or thesis. Because the projects are
so different from one another, there is little we can offer
in the way of general direction. You and your
advisor\index{advisor!role of} will
decide the course of the project. In the spring semester,
you will work closely, and meet regularly, with your advisor
to set intermediate goals, review progress, and solve any
problems that arise. You will also attend group meetings,
approximately monthly, to share news about progress,
difficulties, and successes with your fellow students.

It is vitally important that you maintain a realistic
schedule. Begin work right away, and try to make real
progress in the first month. With your advisor, set
intermediate goals and review the schedule frequently. As
you work, some tasks will take more time than you expect and
others (a few) may take less. Build into your schedule some
slack for the ``slippages'' that always occur, and allow
enough time for preparing the presentation and writing the
paper.

Throughout your work, remember Hofstadter's
Law~\cite{HofstadterGEB}. We are grateful to Jim Marshall
for directing us to this fundamental truth of the universe.
\index{Hofstadter's Law}
\begin{quote}
\textbf{Hofstadter's Law:} It always takes longer than you
expect, even when you take into account Hofstadter's Law.
\end{quote}\index{scheduling|)}


%%%%%%%%%%%%%%%%%%%%%%%%%%%%%%%%%%%%%%%%%%%%%%%%%%%%%%%%%%%%
%
% Chapter: Presentation
%
%%%%%%%%%%%%%%%%%%%%%%%%%%%%%%%%%%%%%%%%%%%%%%%%%%%%%%%%%%%%
\chapter{Presentation}
\label{Chapter:Presentation}
\index{presentation|(}
The project presentations occur late in the spring
semester. Each student will give a presentation on one of 
the dates listed in
\namedref{Appendix}{Appendix:Schedule}; the exact times
and places will be announced. Each student will have
30~minutes: 20~minutes for the presentation and 
10~minutes for questions.


\section{Planning and Rehearsal}
\index{presentation!rehearsing|(}
A good presentation requires careful planning and
judgment. Your intended audience is the group of Computer
Science majors from the junior and senior classes, but
remember that those people have not spent the last few
months intensely involved with your project. Think about how
to introduce your work: what to include and what to omit.

In 20~minutes, you cannot say everything. It is acceptable,
often highly desirable, to concentrate on just one aspect of
your project. Be sure that you provide a clear statement of
the problem and, later in the talk, a clear conclusion. Give
enough important details so that the audience has an
appreciation of your work but not so many that they become
numb.

At the end of the talk, audience members should appreciate
the central ideas and lines of reasoning in your
project. Ideally, they will feel that they could, if they
wanted, fill in the details for themselves.

Once you have an outline, it is time to begin
rehearsing. Practice speaking out loud, while you are
standing up, preferably with someone listening. You will
discover that the presentation takes longer when you
rehearse it in semi-realistic circumstances than when you
whisper it to yourself. You will also discover the points
for which you do not (yet) have good explanations. You are
likely to revise your outline radically after one or two
rehearsals.

After you have established the outline, it is time to refine
the actual presentation. Work to overcome any bad speaking
habits. Avoid speaking too softly or too quickly. Speak
clearly and project to the rear of the room. Maintain eye
contact with various members of the audience. Develop a
relaxed but authoritative conversational tone. Again,
several rehearsals---while you are standing and speaking
aloud to another person---are essential.

You may want to rehearse before a video camera and evaluate
your performance.\index{video equipment}
If so, your advisor\index{advisor!role of} can arrange to
borrow video equipment from ITS.%
\index{presentation!video|see{video equipment}}
\index{presentation!rehearsing|)}


\section{Visual Aids}
\index{presentation!slides|(}
Most students will use some kind of visual aid. Slides of
the Powerpoint variety are most common, but other creative
kinds of demonstrations are welcome.

Keep in mind that at least half of your audience's attention
will be on the visual image and not on what you are
saying. Choose slides or other demonstrations to supplement
your words. If you have a chart, illustration, or animation
that requires some attention, allow the audience time to
absorb it. In the case of an animation or film clip,
particularly, a few seconds of silence from you will be
appreciated.

Most experts suggest that a slide should appear for at least
one full minute. Slides with charts, tables, diagrams, or
illustrations may need more explanation and should be
displayed longer. This means that your 20-minute
presentation should have \emph{at most} 20~slides and
perhaps significantly fewer. The rule is not absolute, and
you are free to include more slides as long as you do it
consciously and for a good reason.

Unless your slide contains a particularly pithy quotation,
do not read from it. Find other words to amplify the idea or
give additional perspective. Above all, \emph{do not read
  bullet points on a slide.}

When they are done well, visual aids greatly enhance a
presentation. Take the time to prepare effective visual
aids.\index{presentation!slides|)}


\section{Logistics}
\index{presentation!setting up|(}
To avoid unpleasant surprises, practice in the actual room
with the actual audio-visual equipment shortly before your
presentation. Make no assumptions! It is possible that
someone will change the settings of the equipment or erase
your file from the computer a few hours before the
presentation. Be prepared for the network to go down just
minutes before you begin.

Some students will have specialized computer demonstrations
that require some work to set up. Please help us to make the
transitions between speakers smooth and fast. Speak to the
people who will appear before and after you and decide how
you will make the shift from one person to the next. Do as
much of the preliminary work as you can before the
presentations begin on that day. If you have very special
needs, your presentation can be scheduled first (or
immediately after a break) to allow time to get ready.%
\index{presentation!setting up|)}
\index{presentation|)}


%%%%%%%%%%%%%%%%%%%%%%%%%%%%%%%%%%%%%%%%%%%%%%%%%%%%%%%%%%%%
%
% Chapter: Prose
%
%%%%%%%%%%%%%%%%%%%%%%%%%%%%%%%%%%%%%%%%%%%%%%%%%%%%%%%%%%%%
\chapter{Prose}
\label{Chapter:Prose}
If you are following the project or research thesis option,
this chapter is for you. It describes the paper or thesis
that will be the permanent record of your work. The written
work is a big part of your senior exercise. A hard copy will
be archived in the Computer Science Department, and
electronic copies will appear on the department web site.

If you are following the clinic option, you may skip this
chapter. The written work is also a big part of your effort,
but you and your teammates will write a report that follows
the specifications of the Harvey Mudd College Computer
Science Clinic.


\section{Content}
\label{Section:Content}
\index{paper and thesis|(}
The purpose of the paper\footnote{To avoid the cumbersome
  phrase ``project paper or research thesis,'' we use the
  word ``paper'' to refer generically to the written work.}
is to explain the endeavor: its background, its goals, and
its outcome. Normally, there will be an introductory
chapter, a background chapter (which was drafted as the
literature review in the fall semester), one or more
chapters that describe the results and observations, and a
conclusion. Often there will be a section on ``lessons
learned'' or ``future work.'' The actual organization will
be determined by the nature of the work. Discuss the
structure of your paper with your
advisor\index{advisor!role of}.

Concentrate on the \emph{results} and use them to guide the
organization of the paper. Once you know where you are
going, you can start at the beginning with the historical
material and make your way to your conclusion. Write clearly
and specifically. Know what you want to say before you begin
to write. Have a concrete conclusion, and include any 
details you need to support it. Present specifications,
derivations, and statistical analyses whenever they are
appropriate. Avoid vague generalities. Set aside time
for composition. It takes work to explain a technical topic
clearly and economically.

The paper should be written for other Computer Science
majors: people who have a solid background in the subject
but who may not know anything about the specific topic. It
is especially difficult for someone who has been working on
a topic for months to back off and write about the most
basic and general facts. The introduction, particularly,
will require patient writing and rewriting to explain the
context without being tedious or long-winded.

Many students ask, ``How long must a project paper be?''%
\index{paper and thesis!length} The
answer is, ``Not one word longer---or shorter---than it
takes to explain the background, the project, and the
results.'' It depends on the subject. If the project is
expository or formulates concepts, the paper will be
longer. If the project's goal is to create something, like a
program or a robot, then the paper may be shorter. Most
project papers are between 20 and 50 pages of text, not
counting the front matter, appendices, and
bibliography. (For comparison, this guide has
\plainpageref{EndOfText}~pages of text.) A typical research
thesis is longer, but there is a wider variation among
theses.

Programs or program fragments \emph{may} appear in a paper
if they are necessary for the exposition. Often, however,
you can make your point just as clearly by using
pseudo-code---or without reference to code at all.  A full
code listing is not required, even though we illustrate one
in \namedref{Appendix}{Appendix:Source}.  In some cases an
advisor will ask that the code be submitted separately from
the paper.

Most papers and theses will not contain a diary of daily
progress.  There is seldom a need for a chronography of what
you did each day. The reader is interested in what you
learned and why, and not in a play-by-play description of
how you did it. There are a few instances, however, in which
the direct experience is crucial---in projects that explore
techniques of software engineering, for example---and if
yours is such a project, you should describe the process.


\section{Form}
\label{Section:Form}
\index{paper and thesis!form|(}
Your final document will be formatted like this guide. Most
of the remainder of this chapter describes how to do it. We
will use a customized style in the \LaTeX\ document
preparation system. You will have only a few decisions about
the format of the paper so that you may concentrate on the
prose and the
content.

We expect good composition, correct spelling, and proper
punctuation. Use a style manual whenever there is any
question. You probably have one from your \emph{Critical
Inquiry} course in your first year, either
\cite{DHackerReference} or \cite{DHackerRules} by Diana
Hacker. \textit{The Elements of Style} \cite{StrunkWhite} by
Strunk and White is another standard style guide. The
\textit{Chicago Manual of Style} \cite{ChicagoStyle} is
comprehensive, scholarly, and authoritative. \textit{A
Handbook for Scholars} \cite{HandbookForScholars} is a
shorter work that is widely used by computer scientists, and
the anthology \textit{Mathematical Writing}
\cite{KnuthMathWriting} has many good
suggestions. Familiarize yourself with some of these books
before you begin writing.

Find someone else---perhaps several people---to help you
with proofreading throughout the writing process. Have
someone read an early draft to show you where you are not
clear. Later, have someone read a nearly-final draft to
catch mechanical errors.

Use a spell-checker.\index{spell check}
The program \texttt{ispell} is
installed on the Computer Science cluster, and it can be
accessed from inside \texttt{emacs} where it will avoid many
of the \TeX-isms. Other systems have similar tools.

Be sure to give credit for previous work and to cite sources
correctly.  Most likely, you will have citations throughout
your paper, and in particular, the background chapter will
be dense with citations. The style of citation is described
below, in \namedref{Section}{Subsection:Citations}.

When you are finally finished, please send copies of the
completed paper to your advisor and to the instructor in
charge of the senior exercise by the announced due date.%
\index{paper and thesis!form|)}%
\index{paper and thesis|)}


\section{A Brief Introduction to \LaTeX}
\label{Section:LaTeXIntro}
\index{latex@\LaTeX|(}
\LaTeX\ is a typesetting system that produces sophisticated
documents. It can be very complicated, but we have tried to
keep the details from becoming bothersome. It is a good
idea, though, for you to become familiar with the system
early in the writing process.

\subsection{Sample Files}
The directory \texttt{/common/cs/senior-exercise/latex}%
\index{sample files}
contains a number of files that you will find helpful.
\begin{itemize}
\item
The file \texttt{srpaper.cls}%
\index{srpaper class@\texttt{srpaper} class!class file}
defines the format for
the senior paper; you must place a copy of it in your
working directory. Should you want to investigate the
inner workings of the format, however unlikely that
may be, see \texttt{srpaper-doc.pdf}%
\index{srpaper class@\texttt{srpaper} class!documentation}
for the annotated source.
\item
For the annotated bibliography, the file
\texttt{annbib-template.tex}%
\index{annotated bibliography!sample files}%
\index{sample files!annotated bibliography}
is the place to start. It gives
you the general framework for the annotated bibliography.
\item
For your paper or thesis, the file
\texttt{srpaper-template.tex}%
\index{sample files!paper template}
is a place to start. It gives
you the general framework for a \LaTeX\ file.
\item
The source to this document is \texttt{srexercise.tex}.
It provides examples of many of the constructions that
you will encounter.
\item
The file \texttt{srpaper-sample-biblio.bib}%
\index{annotated bibliography!sample files}%
\index{sample files!annotated bibliography}
is the
bibliography file for this document. It has examples of most
of the kinds of citations, including annotations, that you
will want to use.
\end{itemize}

\subsection{Resources}
\label{Subsection:References}
\index{latex@\LaTeX!references}
The standard reference for \LaTeX\ is~\cite{Lamport1994},
which also contains a good introduction to \BibTeX. A
deeper look is provided by~\cite{Companion}. For low-level
details, see the original description~\cite{Knuth1984} of
\TeX\ by Knuth. Links to on-line resources and other books
can be found at the following sites:
\begin{indented*}
\url{http://www.dci.pomona.edu/docs/tex-dci.html}\\
\url{http://www.latex-project.org}\\
\url{http://www.tug.org}
\end{indented*}
If you are planning to write on a computer other than 
one in the Computer Science cluster, you may want to 
look at the following site which contains information
about using \LaTeX\ on Windows and Macintosh computers.
\begin{indented*}
\url{http://www.dci.pomona.edu/docs/tex.html}
\end{indented*}

There are products, some of them
free, that ease the burden of processing and viewing
\LaTeX\ documents. You may want to investigate them.

\subsection{Running \LaTeX}
\index{latex@\LaTeX!running}
Some people execute \LaTeX\ in a terminal window,
as described below. Others prefer to use an interface
program. The Macintosh computers in the Computer Science
Laboratories have an easy-to-use interface program called
TeXShop. If you want to install your own system, there are
several different packages for each of the popular
operating systems. They should all produce identical
documents.

If you are new to \LaTeX, start by recreating this
document. On the Computer Science systems, copy these five
files into the directory where you plan to keep your
paper.\index{srpaper class@\texttt{srpaper} class!class file}%
\index{sample files!this document}
\begin{code}
srexercise.tex
srpaper.cls
srpaper-sample-biblio.bib
pclogo-4c-noborder.pdf
\end{code}
Then simply type the command below.
\begin{code}
pdflatex srexercise
\end{code}
You will create a PDF file named
\texttt{srexercise.pdf}.
If you are working on another system, the process may
be a little different.

To begin your own annotated bibliography,
\index{annotated bibliography!formatting}%
\index{bibtex@\BibTeX!running} start with the
file \texttt{annbib\discretionary{-}{}{-}template.tex}.%
\index{annotated bibliography!template} Copy
it into your working directory, rename it, and edit it. Then
use the commands \texttt{pdflatex}, \texttt{bibtex}, and
\texttt{pdflatex} (twice) to create the first draft of your
bibliography. See \namedref{Sections}{Subsection:Citations}
and~\ref{Subsection:Bibliographies} for more information on
citations and bibliographies in \LaTeX.

To begin your own literature review or paper,
\index{literature review!formatting} start with the
file
\texttt{srpaper\discretionary{-}{}{-}template.tex}%
\index{sample files!paper template}. Copy it
into your directory, rename it, and edit it. Then use the
command \texttt{pdflatex} to create the first draft.


\section{Anatomy of a \LaTeX\ File}
\label{Section:LaTeXAnatomy}
A \LaTeX\ file can be created in any text editor. It is
best, however, to use an editor that saves the file as plain
text. We suggest that you avoid word processors because they
often insert formatting information and non-standard
characters.

The comment character is the percent sign \verb|%|. 
A comment extends to the end of the current line. 

You may want to follow along with the template or the
source of this document as you read the following
descriptions of the parts of a file.

\subsection{The Class and Package Specifications}
\label{Subsection:ClassAndPackage}
The first non-comment line of a \LaTeX\ file is the
\texttt{documentclass} specification. There are standard
classes, including \texttt{article}, \texttt{letter},
\texttt{report}, and \texttt{book}. We will use a custom
version called
\texttt{srpaper}.%
\index{srpaper class@\texttt{srpaper} class!class file}
\begin{vcode}
\documentclass[draftcopy]{srpaper}
\end{vcode}
The options appear between the brackets. Here,
\texttt{draftcopy} is the only one.  For your final copy,
you \emph{must} change \texttt{draftcopy} to
\texttt{finalcopy}. That will change the pagination so that
sections start on right hand pages and remove the header
``Draft of $\langle$\textsf{date and time}$\rangle$.'' See
\namedref{Section}{Subsection:Options} for the other options
that are available.

After the class specification comes the (optional) package
specifications. The packages add additional features to
\LaTeX\ and control the appearance of the output. Here are
the specifications for two common packages which are used in
the creation of this document.
\begin{vcode}
\usepackage{url}
\usepackage{graphicx}
\end{vcode}
As you write, you may want to add other packages to the
list. See \namedref{Sections}{Subsection:Packages}
and~\ref{Subsection:FormattingPrograms} for some commonly
used packages. Do not, however, do anything to change the
style or organization specified by \texttt{srpaper}.

\subsection{The Front Matter}
The \texttt{srpaper}%
\index{srpaper class@\texttt{srpaper} class!front matter}
class automatically produces a title
page, abstract, and table of contents, plus optional
acknowledgments, list of figures, and list of
tables. The specifications for this material appears right
after the class and package specifications.

The date must be of the form ``Month dd, yyyy.'' The name(s)
of your advisors\index{advisor!title page format} must be
followed with a comma and either ``advisor'' or
``advisors,'' as appropriate.

Here is a minimal example of the front matter specification,
with title, author, date, advisor, and abstract. When the
acknowledgment is missing or empty, as it is here, the
acknowledgment page will not appear.
\begin{vcode}
\title{A Guide through the Senior Exercise}
\author{Everett L. Bull, Jr.}
\date{July 10, 2006}
\advisor{Professors Kim Bruce, Tzu-Yi Chen,
    and Sara Sood, advisors}
\abstract{A senior exercise ... }
\acknowledgment{}
\end{vcode}

\subsection{The Body}
Most of your paper will be ordinary text---just type it
normally between the delimiters \verb|\begin{document}| and
\verb|\end{document}|. The first line after
\verb|\begin{document}| must be \verb|\frontmatter| to 
get the title page, table of contents, and other preliminary
material. A blank line divides paragraphs.

\subsubsection*{Punctuation}
\index{latex@\LaTeX!punctuation}
The most common punctuation marks---commas, colons, question
marks, and the like---can be typed on the keyboard. Two
matters of punctuation deserve mention, though. Quotation
marks require two keystrokes each: a pair
\texttt{\symbol{'022}\symbol{'022}} of grave accents to open
the quotation and and a pair
\texttt{\symbol{'015}\symbol{'015}} of apostrophes to close
it. Do not use the double quotation mark \texttt{"} from the
keyboard. Some editors, including some configurations of
\texttt{emacs}, will automatically translate the double
quotation mark.

The keystroke \verb|-| produces a hyphen, the punctuation
mark used to break words across line boundaries and to
construct compound words like ``user-friendly.'' Two hyphens
produce an en dash, used in ranges like ``pages~29--47'' and
2006--2007. Three hyphens produce an em dash---the
punctuation mark used to set off or emphasize clauses. In
math mode, a single hyphen produces a minus sign.

The keystroke \verb|~| produces a non-breaking space. It is
exactly like a normal inter-word space, except that a
line-break will not occur at that point. It is normally used
to connect a name and a number, as in \verb|page~47|.

\LaTeX\ offers a full range of punctuation marks and
accents; see the \LaTeX\ documentation for details.

\subsubsection*{Fonts}
For \emph{emphasis,} use the \verb|\emph| command:
\begin{code}
For \verb|\emph{emphasis,}| use \ldots
\end{code}
To use italics for other purposes, the titles of books for
example, use the font-changing command \verb|\textit|:
\begin{code}
Brooks wrote \verb|\textit{The Mythical Man-Month}| \ldots
\end{code}
\namedref{Table}{Table:FontChanges} shows the other commands
for changing the shape or weight of the font.
\begin{table}
\begin{center}
\begin{tabular}{|l|l|}
\hline
\textit{italics}     & \verb|\textit| \\ \hline
\textbf{bold}        & \verb|\textbf| \\ \hline
\textsf{sans serif}  & \verb|\textsf| \\ \hline
\texttt{mono-spaced} & \verb|\texttt| \\ \hline
\end{tabular}
\end{center}
\caption[Font-changing Commands]{The standard
  \LaTeX\ font-changing commands.}
\label{Table:FontChanges}
\end{table}

\subsubsection*{Sectioning}
Use the directives \verb|\chapter|, \verb|\section|, and
\verb|\subsection| to organize your paper.
\begin{vcode}
\subsection{The Body}
Most of your paper ...
\end{vcode}
There are also subsubsections, paragraphs, and
subparagraphs.  To suppress the automatic numbering, put an
asterisk after the directive. The heading above was created
with \verb|\subsubsection*{Sectioning}|.


\section{Further Details about \LaTeX}
\label{Section:LaTeXDetails}
We do not give a comprehensive description of
\LaTeX. Instead, this section touches on some of the
problems you are likely to encounter and gives hints at
their solutions. See the references mentioned in
\namedref{Section}{Subsection:References} for more detailed
information.

\subsection{Options}
\label{Subsection:Options}
\index{srpaper class@\texttt{srpaper} class!options|(}
The options to the \texttt{srpaper}
class appear, separated
by commas, between the brackets in the \verb|\documentclass|
declaration. Here are the possible options for the class.
\begin{description}
\item[\texttt{finalcopy} \textnormal{and}
  \texttt{draftcopy}] control the pagination. With the
  \texttt{finalcopy} option, the text is single spaced with
  each section starting on a right-hand page. The
  \texttt{draftcopy} option avoids blank pages by not
  skipping left-hand pages and puts a heading ``Draft of
  $\langle$\textsf{date and time}$\rangle$'' at the top of
  each page. Only one of the two options may be used. The
  default is \texttt{finalcopy}.
\item[\texttt{short}] is used for short papers and
  fragments. Use it, for example, when you are asked to
  submit ``one chapter'' of your paper. It omits most of the
  front matter, provides a simple title, and does not start
  a chapter on a new page.
\item[\texttt{singlespace}\textnormal{,}
  \texttt{onehalfspace}\textnormal{, and}
  \texttt{doublespace}] control the inter-line spacing. As
  the names indicate, \texttt{onehalfspace} and
  \texttt{doublespace} increase the vertical distance
  between lines of text. The
  front matter is always single-spaced. The
  \texttt{onehalfspace} and \texttt{doublespace} options
  also make the side margins smaller---and the lines
  correspondingly wider.  Only one of the three options may
  be used. The defaults are \texttt{doublespace} for
  \texttt{draftcopy} and \texttt{singlespace} for
  \texttt{finalcopy}.
\item[\texttt{lof} \textnormal{and} \texttt{nolof}]
  determine the presence of a list of figures. Only one of
  the two options may be used, and both are ignored in the
  presence of \texttt{short}. The default is \texttt{lof}.
\item[\texttt{lot} \textnormal{and} \texttt{nolot}]
  determine the presence of a list of tables. Only one of
  the two options may be used, and both are ignored in the
  presence of \texttt{short}. The default is \texttt{lot}.
\item[\texttt{cm} \textnormal{and} \texttt{mathtime}] are
  the font options. The \texttt{cm} option gives the
  standard Computer Modern fonts for \TeX. The
  \texttt{mathtime} option gives Times Roman text with
  matching math fonts. Special files are required for
  \texttt{mathtime}; see below. Only one of the two options
  may be used. The default is \texttt{cm}. If you choose
  the Times Roman fonts, you must copy all the files from
  the directory
  \url{/common/cs/senior-exercise/latex/belleek}
  into your working directory.
\end{description}
The documents submitted to the department must be created
with the \texttt{finalcopy} option.%
\index{srpaper class@\texttt{srpaper} class!options|)}

\subsection{Packages}
\label{Subsection:Packages}
\index{latex@\LaTeX!packages}
There are more packages for \LaTeX\ than you can imagine,
and you will find some of them useful. You can ``use'' them
at the top of your file, with the \verb|\usepackage|
declaration, as described in
\namedref{Section}{Subsection:ClassAndPackage}. You are free to
use any package you like, as long as it does not change the
overall appearance as defined in the \texttt{srpaper}
class. We describe a few examples of packages here and some
others in \namedref{Section}{Subsection:FormattingPrograms}.

The \texttt{graphicx} package gives you the ability to
include graphics.\index{latex@\LaTeX!graphics} See
\namedref{Section}{Subsection:Graphics}. The
\texttt{graphicx} package is used in this document.

The \texttt{url}\index{latex@\LaTeX!formatting urls}
package adds the facility to format
url's. The \texttt{url} package is also used in this
document. 

The \texttt{verbatim}%
\index{latex environments@\LaTeX\ environments!verbatim@\texttt{verbatim}}
package adds the ability to include a
file ``verbatim''---in typewriter font. It is useful for
code listings, as in 
\namedref{Appendix}{Appendix:Source}. One can
even call in a source code file without retyping or cutting
and pasting.  The \texttt{alltt}%
\index{latex environments@\LaTeX\ environments!alltt@\texttt{alltt}}
package is a variant on
\texttt{verbatim} that is a little more flexible. It uses
the typewriter font and respects line breaks, but it allows
some of \LaTeX's formatting capability. Both
\texttt{verbatim} and \texttt{alltt} are automatically
included by the \texttt{srpaper} class.

The \texttt{amssymb} package adds many of the less-common
mathematical symbols.%
\index{latex@\LaTeX!mathematical notation}

\subsection{Environments}
\index{latex environments@\LaTeX\ environments}
A \LaTeX\ environment is bracketed by the delimiters
\verb|\begin{|\texttt{\textsf{envname}}\verb|}| and
\verb|\end{|\texttt{\textsf{envname}}\verb|}|, where
\textsf{envname} is the name of the environment.  A bulleted
list is created with the \texttt{itemize} environment.%
\index{latex environments@\LaTeX\ environments!itemize@\texttt{itemize}}%
\index{latex@\LaTeX!bulleted lists}
\begin{vcode}
\begin{itemize}
   \item This is the first item in the list.
   \item This is the second.
\end{itemize}
\end{vcode}
This code produces the list below.
\begin{itemize}
\item This is the first item in the list.
\item This is the second.
\end{itemize}
A numbered list is created in the same way, except that the
environment is named \texttt{enumerate}.%
\index{latex environments@\LaTeX\ environments!enumerate@\texttt{enumerate}}%
\index{latex@\LaTeX!numbered lists} The 
\texttt{itemize} and \texttt{enumerate} environments may be
nested within one another.

Descriptions and glossaries are created with the
\texttt{description} environment.%
\index{latex environments@\LaTeX\ environments!description@\texttt{description}}
The list of options in
\namedref{Section}{Subsection:Options} was created with the
\texttt{description} environment. The \texttt{quote}%
\index{latex environments@\LaTeX\ environments!quote@\texttt{quote}}
environment is handy for displaying indented blocks of
text. The \texttt{center}%
\index{latex environments@\LaTeX\ environments!center@\texttt{center}}
environment does not create a list
at all; it simply centers text on a line.

There are four additional environments that are provided by
the \texttt{srpaper} class. The environment
\texttt{indented}%
\index{latex environments@\LaTeX\ environments!indented@\texttt{indented}}
simply displays a block of text with the
left margin shifted to the right. The environment
\texttt{indented*}%
\index{latex environments@\LaTeX\ environments!indented*@\texttt{indented*}}
does the same thing and single spaces the
text. If the surrounding text is already single spaced, then
\texttt{indented} and \texttt{indented*} behave identically.

The environment \texttt{code}%
\index{latex environments@\LaTeX\ environments!code@\texttt{code}}
indents and shifts to the
typewriter font. It is an \texttt{alltt} environment nested
inside an \texttt{indented*} environment. The \texttt{vcode}%
\index{latex environments@\LaTeX\ environments!vcode@\texttt{vcode}}
environment is the same, except that the inner environment
is \texttt{verbatim}. The fragment showing the
\texttt{itemize} environment, above, is set using a
\texttt{vcode} environment.

\subsection{Formatting Programs}
\label{Subsection:FormattingPrograms}
\index{latex@\LaTeX!formatting programs|(}
It is difficult to display programs and code fragments in an
effective and pleasing way. The most common problems in
\LaTeX\ are indentation, long lines, and reserved characters
like \verb|{|, \verb|}|, and \verb|\|. See the source to
this document for some simple examples of short blocks of
code. The construction construction that produces
\namedref{Figure}{Figure:Fibonacci} is particularly awkward
because we want the code centered in the figure.
\begin{figure}
%
% A note on the "awkward construction":
%
% The trick here is to create the displayed material and
% then center it. We do this by putting it inside a
% tabular environment and switching explicitly to the
% monospaced font. With the code, vcode, or verbatim
% environments, the program would be left-justified
% instead of centered.
%
\begin{center}\ttfamily
\begin{tabular}{l}
fun fib k = if k < 2 \\ \obeyspaces
    then max(0,k)    \\ \obeyspaces
    else fib (k-1) + (fib (k-2));
\end{tabular}
\end{center}
\caption[The Fibonnaci function]{The Fibonacci function in ML.}
\label{Figure:Fibonacci}
\end{figure}

For short fragments of code, you can use the \texttt{alltt}
or \texttt{verbatim} environments, or their indented
counterparts \texttt{code} and \texttt{vcode}. For longer
programs, say a full listing in an appendix, the
\verb|\verbatiminput| command from the \texttt{verbatim}
package is handy. For an example, see
\namedref{Appendix}{Appendix:Source}.

If you have formatting problems, you might look at the
packages \texttt{alg}, \texttt{algorithmic},
\texttt{fancyvrb}, and \texttt{listings}. The last of these
appears to parse the programming language and format it
accordingly.\index{latex@\LaTeX!formatting programs|)}

\subsection{Formatting Mathematics}
\index{latex@\LaTeX!mathematical notation}
\TeX\ and \LaTeX\ were originally created for typesetting
mathematics. You can include formulas like $E=mc^{2}$ and
displayed equations like
\[
   \frac{\pi}{4}=\sum_{i=0}^{\infty}(-1)^{i}\frac{1}{2i+1}.
\]
Look at the source for this document (in
\namedref{Appendix}{Appendix:Source}) to see how these formulas
were created. See \cite{Lamport1994}, \cite{Knuth1984}, and
the on-line references for more about formatting
mathematics.

\subsection{Managing a Large Document}
\chardef\bslash=`\\
\newcommand{\cs}[1]{\texttt{\bslash{}#1}}
It is cumbersome to maintain a large document in a single
file. You may split up your paper into several sections,
each one in a different file. Use the \verb|\input|
directive to insert the contents of a file at
the appropriate place.\footnote{We use plain \TeX's \cs{input}
instead of the more sophisticated \LaTeX\ \cs{include}
because the latter always starts a new page and can cause
unwanted page breaks.} Frequently, there is a master file
that contains only front matter and \verb|\input|
directives.
\begin{vcode}
\input section4.tex
\end{vcode}

\subsection{Labels and References}
\label{Subsection:LabelsAndRef}
\index{latex@\LaTeX!cross references|(}
\LaTeX\ labels provide an easy way to cross-reference
material. Think of a label as an anchor. Inserting the
directive \verb|\label{name}| at some point creates a
reference to the material at that point and records the page
number where it appears. The reference can be an item number
in an enumerated list, a section or subsection number, or
the number of a figure or table. Normally the \verb|\label|
directive appears right after the declaration of the entity
to which it refers. Figures and tables are a little
different; the \verb|\label| appears at the end, after the
caption.

To use a label, simply type \verb|\ref{name}|, where
``name'' is the name that was declared with the label. You
will find it easiest, in the long run, to use descriptive
names. To refer to a page, type \verb|\pageref{name}|.
Obviously, there can be only one \verb|\label|
declaration for a given name, but there can be many
references to it.

Here is an example that produces the sentence
``Section~\plainref{Subsection:LabelsAndRef} begins on
page~\plainpageref{Subsection:LabelsAndRef}.''
\begin{vcode}
\subsection{Labels and References}
\label{Subsection:LabelsAndRef}
...
Section~\ref{Subsection:LabelsAndRef} begins on
page~\pageref{Subsection:LabelsAndRef}.
\end{vcode}
Recall that the \verb|~| symbol is a non-breakable space 
used to prevent a line break between the word ``Section''
and the number that follows it.

You may have to process the file twice before the labels are
properly resolved. Pay attention to error messages and
warnings about duplicate or non-existent labels.%
\index{latex@\LaTeX!cross references|)}

\subsection{Figures and Tables}
\index{latex@\LaTeX!figures and tables|(}
You will almost certainly want to use some sort of figure
or table in your paper. The general framework is like
this:%
\index{latex environments@\LaTeX\ environments!figure@\texttt{figure}}
\begin{code}
\verb|\begin{figure}|
\qquad \textrm{\ldots figure specification \ldots}
\verb|\caption[|\textrm{short title, for the list of figures}\verb|]{|\textrm{text of caption}\verb|}|
\verb|\label{Figure:Name}|
\verb|\end{figure}|
\end{code}
The order is important: first the figure, then the caption,
and finally the label.

Tables are similar with the word \texttt{table}%
\index{latex environments@\LaTeX\ environments!table@\texttt{table}}
replacing
\texttt{figure}. The material for a table is usually
formatted with the \texttt{tabular} environment. The source
for \namedref{Table}{Table:FontChanges} is a simple example.

\subsubsection{Positioning Figures and Tables}
\index{latex@\LaTeX!positioning figures and tables|(}
Figures and tables are called ``floating matter'' because
they float among the paragraphs of the text. \LaTeX's 
algorithm for positioning floating matter is complicated
and sometimes unpredictable. Often the figure or table 
appears in an unwanted position.

A small change to the text---even a line or two---can
drastically change the placement of figures or tables.
Do not worry about the positions of figures and tables
until you are ready to produce the final copy. (But
do leave yourself enough time at the end to attend to
the details of visual formatting!)

The first step in controlling placement is to use
the placement option in the \texttt{figure} or
\texttt{table} environment. The options are \texttt{t},
\texttt{b}, \texttt{h}, and \texttt{p}, standing
respectively for top, bottom, here, and ``on a
separate page.'' The options appear in square brackets
after the opening of the environment.
\begin{code}
\verb|\begin{figure}[t]|
\end{code}
Understand, however, that these options are merely
suggestions; \LaTeX\ has a number of other constraints.
Sometimes it will help to add an exclamation mark, as
if to say ``I really mean it!''
\begin{code}
\verb|\begin{figure}[t!]|
\end{code}
If that does not solve the problem, there are stronger
measures. Read about them in Chapter~6
of~\cite{Companion}.%
\index{latex@\LaTeX!positioning figures and tables|)}

\subsubsection{Large Figures and Tables}
\index{latex@\LaTeX!large figures and tables|(}
You may find that your figure or table will not fit within
the margins, especially when using the \texttt{finalcopy}
option. In those rare cases, it is permissible to have
a figure or table that is wider than the text. One easy
way to accomplish that is to use the \texttt{adjustwidth}
environment, available with the \texttt{changepage}%
\index{latex environments@\LaTeX\ environments!changepage@\texttt{changepage}}
package. Look for the \texttt{changepage} documentation
if you have a need for it.

A more drastic step is to print a wide table in 
landscape mode on a page by itself. The \texttt{rotating}%
\index{latex environments@\LaTeX\ environments!rotating@\texttt{rotating}}
package provides commands \verb|\sidewaysfigure| and
\verb|\sidewaystable|.%
\index{latex@\LaTeX!large figures and tables|)}%
\index{latex@\LaTeX!figures and tables|)}

\subsection{Graphics}
\label{Subsection:Graphics}
\index{latex@\LaTeX!graphics|(}
Very likely, you will want to include pictures or diagrams
in your paper. You can do that with the \texttt{graphicx}
package. The Pomona College logo was included in
\namedref{Figure}{Figure:PCLogo} with the command
\begin{vcode}
\includegraphics[width=0.5in]{pclogo-4c-noborder.pdf}
\end{vcode}
For \texttt{pdflatex}, the external graphics files must be
PDF files. You will need to convert other kinds of graphics
to PDF. Other variants of \LaTeX\ allow other types of
files.
\begin{figure}
\begin{center}
\leavevmode
\includegraphics[width=0.5in]{pclogo-4c-noborder.pdf}
\end{center}
\caption[The Pomona College logo]{The Pomona College logo,
  as an example of included graphics.}
\label{Figure:PCLogo}
\end{figure}
The book \cite{GraphicsCompanion} is a comprehensive
reference for graphics in \LaTeX.%
\index{latex@\LaTeX!graphics|)}

\subsection{Footnotes and Citations}
\label{Subsection:Citations}
Footnotes\index{latex@\LaTeX!footnotes} are created with
\verb|\footnote|.\footnote{Remember, though, that footnotes
  are distracting. Work to minimize the number of them.}
For the senior paper we do not use footnotes or
endnotes for citations.%
\index{latex@\LaTeX!citations}%
\index{bibtex@\BibTeX!citations}
 Rather, the citations are given in
the text in brackets. You generate a citation with
\verb|\cite{key}|, where \texttt{key} is a name for the work
cited. Here are a few examples of using \verb|\cite|.
\begin{code}
In\verb|~\cite{Cpp}|, Stroustrup defined the language \ldots
Stroustrup\verb|~\cite{Cpp}| defines the language \ldots
The language C++ was defined by Stroustrup\verb|~\cite{Cpp}|.
\end{code}
The result is the following.
\begin{indented}
In \cite{Cpp}, Stroustrup defined the language \ldots \\
Stroustrup~\cite{Cpp} defines the language \ldots \\
The language C++ was defined by Stroustrup~\cite{Cpp}.
\end{indented}
Remember that a citation at the end of a sentence goes
\emph{before} the punctuation.

\subsection{Bibliographies}
\label{Subsection:Bibliographies}
\index{bibtex@\BibTeX|(}
\index{latex@\LaTeX!bibliography|(}
The citation keys are associated to books and articles in a
bibliography file. The file 
\texttt{srpaper-sample-biblio.bib} is an example of such a
file. Here is a typical entry in that file.
\begin{vcode}
@book{Cpp,
  author = "Bjarne Stroustrup",
  title = "The {C++} Programming Language",
  edition = "third",
  year = 1997,
  publisher = "Addison-Wesley",
  annotate = "The bible for a complex programming
              language. A required reference book
              for anyone doing serious C++ programming."
}
\end{vcode}
The entry describes the book corresponding to the key
\texttt{Cpp}.  Keys are not case-sensitive. The entry
provides the usual bibliographic information. The sample
file contains example entries for articles and other kinds
of documents.

The fields in a bibliographic entry are determined by the
bibliographic style you are using---a choice has been made
for you by the \texttt{srpaper} class. If you add fields of
your own creation, they will be silently ignored. Look at
the examples, or ask your advisor, if you have an item that
is not accommodated by the standard fields.

The entries in a bibliography are formatted automatically by
\LaTeX\ and \BibTeX. All you have to do is provide
the information in the correct form. The content of the
\texttt{annotate} field is displayed when you create an
annotated bibliography but not when you create an ordinary
bibliography. (Technical detail: If you wish to have
multi-paragraph annotations, you must use \cs{par} to create
paragraph breaks.) See the samples files
\texttt{annbib-template.tex} and
\texttt{srpaper-template.tex}.

As you will see in the sample documents, the
\texttt{srpaper} class provides two commands for creating
bibliographies. The command \verb|\bibliography| creates an
ordinary bibliography, like the one at the end of the
document. The \texttt{annotate} field is ignored. The
command \verb|\annotatedbibliography| includes the
annotations, as in the sample \texttt{annbib-template.tex}.

Only the works that you cite in the document will appear in
the bibliography. Although some style guides will tell you
that a bibliography must contain \emph{only} cited works, it
is sometimes desirable to include works that are not
cited. The annotated bibliography is an example of such a
case. To include a single work that is not cited use a
command like the following somewhere in the document.
\begin{vcode}
\nocite{MythicalManMonth}
\end{vcode}
To include \emph{all} the works in a bibliography file, use
the wild-card command.
\begin{vcode}
\nocite{*}
\end{vcode}

The process for connecting a document to a bibliography is a
little involved. First process the document with
\texttt{pdflatex}. Then process the document again with
\texttt{bibtex}. Then run \texttt{pdflatex}
again---twice. The idea is that \LaTeX\ gathers the keys for
a citation on the first run. Then \BibTeX\ takes the
keys and formats the matching bibliography entries. \LaTeX\
collates the entries on the second run and gets everything
right on the third. If you ever add a citation or change the
entry in the bibliography file, you have to repeat the
cycle.

Whenever possible, refer to published journals and not to
web pages. If the web page is all you have, use the format
of \cite{AKS2002}. (Interestingly, that link is no longer valid.
It illustrates the problems with changing web sites.)
If a published paper is also available on
the web, it is helpful to include the url---as was done in
\cite{AKS2004}.%
\index{bibtex@\BibTeX|)}
\index{latex@\LaTeX!bibliography|)}

\subsection{Warnings and Error Messages}
\index{latex@\LaTeX!warnings}
\index{latex@\LaTeX!errors}
The messages that \LaTeX\ displays as it runs are not always
obvious, but you can usually decipher them. There is also a 
log file, with the extension \texttt{.log} that contains even
more information.

The most common issues are overfull or underfull boxes and
missing or duplicate labels. A box that is overfull by a tiny
amount is perhaps not a problem, but one that runs off the
page is unacceptable. All labels and citations should be
correctly resolved.

When you are preparing a final copy to submit, pay attention
to the warnings. Look carefully
at the document and find the text that produced a warning. Make
adjustments so that the paper appears exactly as you want it.%
\index{latex@\LaTeX|)}%
% A trick to get the total number of pages of text, excluding
% front matter and appendices.
\label{EndOfText}

% Add the following to put more examples in the bibliography.
\nocite{*}


%%%%%%%%%%%%%%%%%%%%%%%%%%%%%%%%%%%%%%%%%%%%%%%%%%%%%%%%%%%%
%
% Appendix: Schedule
%
%%%%%%%%%%%%%%%%%%%%%%%%%%%%%%%%%%%%%%%%%%%%%%%%%%%%%%%%%%%%
\appendix
\chapter{Sequence of Events}
\label{Appendix:Schedule}
\index{calendar|(}
\index{scheduling|(}
Precise due dates will be announced in the Senior Seminar,
over e-mail, and on the departmental web site. You are
responsible for knowing about---and meeting---the various
deadlines. The general schedule below is provided to help
you plan and balance your workload during the senior year.

There will be a few students who graduate in September or
December and cannot follow the usual schedule. Those
students will meet individually with the faculty and create
alternate timelines.


\section*{Junior Year}
Students should obtain permission for the research thesis
option or the clinic option early in the second semester.


\section*{Senior Year, Fall Semester}
The tasks to be completed in the fall semester are part of
the Senior Seminar. The fall group progress reports%
\index{progress meetings} will be
at the scheduled time of the Senior Seminar. There will, of
course, be other work for the seminar.

The due dates for the group project are not included here.
There will be similar milestones that will be announced by
the project's advisor.

\begin{description}
\item[Mid-September.] Submission of a ranked list of three
possible senior project advisors and
corresponding project ideas.
\item[Late September.] Title, advisor\index{advisor!choice of},
and description due.\index{topic!description due date}
Submit the title of your project, thesis,
or clinic; the name of your advisor; and a one- or
two-sentence description of the undertaking. All students
will do this, even though the information will be new only
for the students following the project option.
\item[Late October.] Annotated
bibliography due.\index{annotated bibliography!due date}
[Proj\-ect and research thesis only.]
\item[Mid-November] Literature review
and extended abstract due.%
\index{literature review!due date}%
\index{extended abstract!due date}
[Project and research thesis only.]
\item[Last day of classes.] Revisions, if
necessary, of literature review and extended abstract due.
[Project and research thesis only.] 
\end{description}

In addition, a student and advisor may agree on additional preparatory
tasks to be completed during the semester.


\section*{Senior Year, Spring Semester}
The schedule common to all students appears below. In addition
to these activities, you and your advisor\index{advisor!role of}
will agree on dates for activities---like partial drafts and
presentation rehearsals---that are specific to your particular
project.
\begin{description}
\item[Throughout the first half of the semester.]
Prog\-ress report meetings, at three or four week intervals
starting in the first or second
week of classes.\index{progress meetings}
Each student will give an informal, five-minute
presentation on the project's status.
Students in the group project will participate in these
meetings \emph{and} the much more frequent meetings of
their project team.
\item[Mid-April.] \emph{Complete} drafts
of papers and theses due.\index{paper and thesis!drafts}
Drafts of individual chapters will have been submitted
earlier.
[Proj\-ect, group project, and research thesis only.]
%\item[Late April, usually on a Thursday and a Friday.]
% hack to get good spacing with such a long label
\item[]\hspace*{-\labelsep}\textbf{Sometime in April, usually on a Thursday
and a Friday.}\hspace{\labelsep}%
Presentations.
The dates and times will be advertised.\index{presentation!dates}
\item[One week before the end of classes.] Papers and theses due.%
\index{paper and thesis!due date}
[Proj\-ect, group project, and research thesis only.]
\item[Friday after the end of classes.] Senior grades due.
\item[Sunday in mid-May.] Commencement.
\end{description}
\index{calendar|)}
\index{scheduling|)}


%%%%%%%%%%%%%%%%%%%%%%%%%%%%%%%%%%%%%%%%%%%%%%%%%%%%%%%%%%%%
%
% Appendix: Advice and Crisis Management
%
%%%%%%%%%%%%%%%%%%%%%%%%%%%%%%%%%%%%%%%%%%%%%%%%%%%%%%%%%%%%
\chapter{Gratuitous Advice and Crisis Management}
\label{Appendix:CrisisManagement}
Clearly, not every problem can be solved by reading this
booklet. But answers to many common ones \emph{can} be found
here, and we have seen many problems arise when the advice
and directions given here are ignored.

We expect you to read and assimilate every word of this
document within twenty-four hours of receiving it. We
recognize, however, that you may forget one or two
details. Here are some pointers that may be helpful in a
crisis. This list is expected to grow, based in large part
on your experience; please make suggestions.

\begin{description}
\item[Getting started.]
If you are uncertain of which of the three options to
select, read the material in
\namedref{Chapter}{Chapter:Premeditation}.
Take the time to make a deliberate choice
from among the three options. Speak with as many
people---especially faculty members and other
students---as you can, and be certain that you are
comfortable with your selection.
\item[Advisor or topic.]
If you have selected the project option and are casting
about for an advisor or topic, read
\namedref{Section}{Section:AdvisorAndTopic}. Speak with
faculty members and other students. Think, as concretely
as you can, about the nature of the project and how you
will pursue it. Remember that the quality of
the final product is singularly important. 
\item[Annotated bibliography.]
If you are faced with creating an annotated bibliography,
discuss possible sources with your advisor and read
\namedref{Section}{Section:AnnotatedBibliography}. For help
with formatting the bibliography, look at the sample files
\texttt{annbib\discretionary{-}{}{-}template\discretionary{.}{}{.}tex}
and \texttt{srpaper-sample-biblio.bib}, and read about
\LaTeX\ generally in \namedref{Section}{Section:LaTeXIntro}
and about bibliographic details in
\namedref{Sections}{Subsection:Citations}
and~\ref{Subsection:Bibliographies}.
\item[Literature review.]
If you are about to write a literature review, speak with
your advisor about the overall approach. Most likely, you
have read some articles that have introductory sections
reviewing previous work. These (usually) are good
examples. Read
\namedref{Section}{Section:LiteratureReview}. For the
document itself, use the sample file
\texttt{srpaper-template.tex}, but remember to specify the 
\texttt{short} option. Read
\namedref{Sections}{Section:LaTeXIntro}, 
\ref{Section:LaTeXAnatomy}, and~\ref{Section:LaTeXDetails}.
\item[Sloth.]
Have a specific timeline. If you are in the middle of your
senior exercise and you are
behind schedule, take action immediately. Reread
\namedref{Section}{Section:Scheduling}. Assess the situation
and make changes. Be willing to omit part of the project, if
necessary. Ask you advisor to make motivational threats.
\item[Writer's block.]
If you are beginning to write a paper or thesis, use a
divide-and-conquer strategy. Separate the content from the 
form, and separate the content into manageable
chapters. Discuss the organization with your advisor. Begin 
writing the introductory chapters early. Be sure that you
have a clear vision of your conclusion. Read the first few
sections of \namedref{Chapter}{Chapter:Prose}.

Begin writing early; there is never enough
time. You should have a substantial start on your paper
or thesis by the beginning of Spring Break. 
Writing the paper is an integral part of the project
and not something to be saved until everything
else is finished.
\item[Formatting.]
If you are working on formatting your paper or thesis using
\LaTeX, review the sample file
\texttt{srpaper-template.tex}. Read the later sections of 
\namedref{Chapter}{Chapter:Prose}. As problems arise, return to
the relevant sections in this document, refer to the
resources listed in \namedref{Section}{Subsection:References}, or
ask your advisor for further help.

Pay close attention to details (\emph{all} of them!) and
produce a professional-quality document. Among other things,
remember to proofread. Double check your spelling. Do not allow a
line of text to run into the margin or off the edge of a page.
If you do not
have any tables, use the \texttt{nolot} option to suppress
the list of tables after the table of contents.
\item[Stage fright.]
As you plan your presentation, reread the advice in
\namedref{Chapter}{Chapter:Presentation}.
Consider the visual aids. Think about the audience. Rehearse.
Then rehearse some more.
\item[Denial.]
If you are following the clinic option and think that none
of this applies to you, remember that you must give a
presentation. See the previous suggestion.
\end{description}


%%%%%%%%%%%%%%%%%%%%%%%%%%%%%%%%%%%%%%%%%%%%%%%%%%%%%%%%%%%%
%
% Appendix: Source
%
%%%%%%%%%%%%%%%%%%%%%%%%%%%%%%%%%%%%%%%%%%%%%%%%%%%%%%%%%%%%
\chapter{Source for This Document}
\label{Appendix:Source}
This appendix exists to demonstrate how to create a source
listing, should you need to do it. Keep in mind, however,
that the sources are usually not required. Check with your
advisor.

To save paper, we have omitted the actual source
listing. The following line, which would have generated the
listing, was removed.
\begin{vcode}
{\small\displayspacing\verbatiminput{srexercise.tex}}
\end{vcode}
The listing can be found in the directory
\texttt{/common/cs/senior-exercise\discretionary{/}{}{/}latex/srexercise.tex}.


%%%%%%%%%%%%%%%%%%%%%%%%%%%%%%%%%%%%%%%%%%%%%%%%%%%%%%%%%%%%
%
% Bibliography
%
%%%%%%%%%%%%%%%%%%%%%%%%%%%%%%%%%%%%%%%%%%%%%%%%%%%%%%%%%%%%
\bibliography{srpaper-sample-biblio}


%%%%%%%%%%%%%%%%%%%%%%%%%%%%%%%%%%%%%%%%%%%%%%%%%%%%%%%%%%%%
%
% Index
%
%%%%%%%%%%%%%%%%%%%%%%%%%%%%%%%%%%%%%%%%%%%%%%%%%%%%%%%%%%%%
\printindex


\end{document}
