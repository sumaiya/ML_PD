%%%%%%%%%%%%%%%%%%%%%%%%%%%%%%%%%%%%%%%%%%%%%%%%%%%%%%%%%%%%%%%%%%%%%%%%
%
% A template for seniors completing their senior papers in Computer
% Science at Pomona College.
%
% See the sample document---and its source---for more details.
%
% Rett Bull
% June 23, 2006
% modified May 30, 2007 (changed \include to \input)
%
%%%%%%%%%%%%%%%%%%%%%%%%%%%%%%%%%%%%%%%%%%%%%%%%%%%%%%%%%%%%%%%%%%%%%%%%



%%%%%%%%%%%%%%%%%%%%%%%%%%%%%%%%%%%%%%%%%%%%%%%%%%%%%%%%%%%%%%%%%%%%%%%%
%
% The line below is sufficient. When your paper is ready to submit,
% change ``draftcopy'' to ``finalcopy.'' See the sample document for
% other options that can appear between the brackets.
%
%%%%%%%%%%%%%%%%%%%%%%%%%%%%%%%%%%%%%%%%%%%%%%%%%%%%%%%%%%%%%%%%%%%%%%%%
\documentclass[draftcopy]{srpaper}



%%%%%%%%%%%%%%%%%%%%%%%%%%%%%%%%%%%%%%%%%%%%%%%%%%%%%%%%%%%%%%%%%%%%%%%%
%
% Insert here any LaTeX packages you want to use. The graphicx package
% is a common one, it is shown here as a (commented-out) example.
%
%%%%%%%%%%%%%%%%%%%%%%%%%%%%%%%%%%%%%%%%%%%%%%%%%%%%%%%%%%%%%%%%%%%%%%%%
%\usepackage{graphicx}
\usepackage{url}


%%%%%%%%%%%%%%%%%%%%%%%%%%%%%%%%%%%%%%%%%%%%%%%%%%%%%%%%%%%%%%%%%%%%%%%%
%
% Fill in the parts between the braces. You will not be able to process
% your document until (at least some of) the items are present.
%
% Format for date: Month dd, yyyy
%
% Format for advisor: Professor Firstname Lastname, advisor
%                 or: Professors X, Y, and Z, advisors
%
%%%%%%%%%%%%%%%%%%%%%%%%%%%%%%%%%%%%%%%%%%%%%%%%%%%%%%%%%%%%%%%%%%%%%%%%
\title{A Machine Learning Approach to Diagnosis of Parkinson's Disease}
\author{Sumaiya Hashmi}
\date{September 20, 2013}
\advisor{Professor Sara Sood}
\abstract{
I will investigate applications of machine learning algorithms to medical
data, adaptations of differences in data collection, and the use of ensemble techniques.

Focusing on the binary classification problem of
Parkinson's Disease (PD) diagnosis, I will apply machine learning
algorithms to a dataset consisting of voice recordings from healthy
and PD subjects. Specifically, I will use Artificial Neural
Networks, Support Vector Machines, and an Ensemble Learning
algorithm to reproduce results from \cite{MS12} and \cite{GJ70}. 

Next, I will adapt a regression dataset of PD recordings and combine
it with the binary classification dataset, testing various techniques
to consolidate the data including treating the regression data as
unlabeled data in a semi-supervised learning approach. I will determine the performance of
the above algorithms on this consolidated dataset. 

Performance of algorithms will be evaluated using k-fold cross
validation and a confusion matrix. Specificity and sensitivity will be
calculated, as these are of particular importance in medical
diagnosis. I will also determine accuracy, precision, recall, and
F-score.

The expands on past related work, which has used either a regression dataset alone to
predict a Unified Parkinson's Disease Rating Scale score for PD
patients, or a classification dataset to determine healthy or PD
diagnosis. The datasets have not been combined, and the regression set
has not been used to contribute to evaluation of healthy subjects.
}
\acknowledgment{}



%%%%%%%%%%%%%%%%%%%%%%%%%%%%%%%%%%%%%%%%%%%%%%%%%%%%%%%%%%%%%%%%%%%%%%%%
%
% The content of your document goes after \frontmatter and before
% \end{document}. To get you started, we have given a few small
% hints.
%
%%%%%%%%%%%%%%%%%%%%%%%%%%%%%%%%%%%%%%%%%%%%%%%%%%%%%%%%%%%%%%%%%%%%%%%%
\begin{document}
\frontmatter

\chapter{Background}
\label{Chapter:One}
Many medical
decision-making questions can be reduced to binary
classification problems, making medical data an ideal domain for
several machine learning techniques. A few of the most relevant
algorithms, based on past work in this domain, are discussed
below. Their applications to medical data in general, reviewed below,
share many similarities with the specific case of
Parkinson's diagnosis.

\section{Machine Learning techniques}
\subsection{Artificial Neural Networks}
Artificial neurons were first proposed in 1943. Motivated by biological neurons, the artificial neuron received
several weighted inputs and produced an output, based on some
threshold \cite{MP43}. The perceptron model built on this early
work, adding a learning rule to improve the performance of the neural
network \cite{Ros58}. However, the perceptron model was severely limited, unable to
solve non-linearly separable functions such as XOR \cite{MP69}. Eventually, multilayer perceptrons were developed to
address the original perceptron model's shortcomings. 

Artificla neural nets have been widely used across a myriad of
applications. For example, a checkers-playing program used neural networks to train a player \cite{CF01}. The board was encoded as a vector of
available board positions, with values assigned based on whether the
square was empty, taken by a regular piece, or taken by a King. The
neural network consisted of three hidden layers. The first hidden
layer completed
spatial preprocessing, representing each subsquare of the board as a
node, for a total of 91 nodes. The second and third hidden layers had
40 and 10 nodes, respectively. The network outputted a value between
-1 and 1, representing the goodness of the board from the current
player's perspective. The weights for the network were initially
specified through a uniform sample, and several networks played against one
another. The winners were declared `parents', and they generated
`offspring networks' with weights varied by a parameter vector. The
process was repeated for many generations to produce an ideal neural
network.

\subsection{Support Vector Machines}
Support vector machines are binary classifiers that can be applied to linearly
separable datasets. They separate data into classes using a
hyperplane. SVMs can also be used non-linearly by mapping the data
to a higher-dimensional space, thus making the data separable. This
mapping is done by a kernel function. SVMs perform well with large feature
spaces, as long as the data is separable with a wide margin. They also
do well with sparse datasets, as in text classification \cite{Joa98}.

In the absence of large amounts of labeled data, pool-based active
learning can be utilized with SVMs \cite{TK01}. The learning
algorithm has access to a pool of unlabeled data, and is able to
choose a subset of that pool to use as training data. The learner
chooses pool data to use such that the data minimizes the size of
the learner's set of hypotheses,
and brings it closer to a single hyperplane. This approach allows for
using less labeled data.

\subsection{Ensemble Learners}
Ensemble learners combine different machine learning algorithms. There
is no one algorithm that always performs well on all domains, and
ensemble learners are a way of combining the advantages of different
learners. A good ensemble algorithm will be made up of diverse base
learners that have varied strengths. The different learners can be
combined in a number of ways. They can work in parallel on all of the inputs, and their
outputs can be combined in some way. Alternatively, a multistage combination will train the base
learners on different subsets of the input data. For example, the
AdaBoost algorithm
first trains an
initial learner, and then trains subsequent learners on data that the
first learner misclassifies. This way, the weaknesses of each
base-learner are made up for by the next learner \cite{FS95}.

\section{Applications to Medical Data}
Medical diagnosis presents an ideal domain for machine learning
algorithms. A large part of diagnosis falls under pattern recognition,
based on large amounts of data, and ML algorithms are well-suited to
this task. For an algorithm to be effective in this domain, it needs to be able to handle noisy and missing
data, rely on relatively few medical tests, and complement the role of
physicians \cite{Kon01}.
Machine learning algorithms have been applied to a variety of medical
data, some examples of which are outlined below.

\subsection{Self-reported input}
Some diagnoses rely largely on patient-reported information, rather than
biological tests. A prime example of this is diagnosis of mental disorders,
which is based on how a patient's symptoms compare to criteria
outlined in the Diagnostic and Statistical Manual of Mental
Disorders. Symptoms are determined through consultation with a mental
health professional, and are largely reported by the
patient. Automated systems
have been proposed which will produce a diagnosis based on
user-reported information \cite{yap1996}.

\subsection{Clinical Decision Support Systems}
Clinical decision support systems help healthcare professionals make
diagnosis decisions based on patient data. These systems can be
rule-based, in which case they are created with a knowledge base and
a set of rules. Alternatively, they can utilize machine learning to
learn from past data and recognize patterns. Several such
systems have been proposed, including a statistical approach to
diagnosing digestive disorders based on an electronically-administered interview of the
patient \cite{SK84}. However, the use is not yet widespread, in part
due to lack of data availability and to limited adoption of uniform
computer systems \cite{Greenes2007}.

\subsection{EEG and EKG data}
Recordings of electrical activity in the body can be used to diagnose a variety of
disorders. Electroencephalograms (EEGs) are recordings from the brain
and contain a wealth of features that can be used by machine learning
algorithms. A classification algorithm using EEGs was able to diagnose
Alzheimer's Disease with 86.05\% accuracy \cite{Pod12}.

Electrocardiograms (EKGs) are often used to detect arrhythmia, which is any
abnormality of the heartbeat. They can be indicative of heart disease
and other conditions. In 1989, a model was derived from the Cleveland Clinic heart disease data set and
compared it to CADENZA, a Bayesian algorithm. Both models were found
to overpredict heart disease, though this occurred more with CADENZA
\cite{Det89}. 

After that, a novel machine learning approach to diagnosing and
classifying cardiac arrhythmia was presented, called the VF15
algorithm. It used a genetic algorithm to learn feature weights. Then,
each feature voted on a class prediction. The algorithm had a 62\%
accuracy on this task and was found to outperform Naive Bayes \cite{GAD98}.

Another study collected data on ischemic heart disease, including signs and
symptoms, EKG, and scintigraphy. Several algorithms were applied, including Naive Bayes, neural networks,
k-nearest neighbors, and two decision tree algorithms. These were compared to clinicians’ diagnoses. Naive Bayes had the best
sensitivity/recall, whereas clinicians, followed by neural nets, had
the highest specificity \cite{KKG+99}.

More recently, a group compared various machine learning algorithms for arrhythmia
diagnosis based on EKG data, with an emphasis on minimizing false
positives and dealing with noisy data. They used the UCI Machine
Learning Repository Arrhythmia dataset, and highlighted the need to
improve on VF15’s 62\% accuracy. They evaluated a Bayesian artificial neural network classifier as compared to Naive Bayes, decision trees, logistic regression, and neural networks \cite{GMCL05}.

\section{Diagnosing Parkinson's Disease}
Parkinson's Disease (PD) is a degenerative neurological disorder
marked by decreased dopamine levels in the brain. It manifests itself
through a deterioration of movement,
including the presence of tremors and stiffness. There is commonly a
marked effect on speech, including dysarthria (difficulty articulating
sounds), hypophonia (lowered volume), and monotone (reduced pitch
range). 
Additionally, 
cognitive impairments and changes in mood can occur.
Traditional diagnosis of Parkinson's Disease involves taking a
neurological history of the patient and observing motor skills in
various situations. Monitoring progression of the disease over time requires repeated clinic visits by the patient. There is no cure, but pharmacological treatment to
manage the condition includes dopaminergic drugs. 

Speech tests can be used for monitoring Parkinson’s disease, due to
vocal impairment being a common symptom and early indicator. Using an
at-home recording device, such as one developed by Intel for PD
telemonitoring, can conveniently allow PD patients' health to be monitored
remotely. Specified voice recordings can be passed through signal processing algorithms and a
classification and regression tree to predict a rating on the unified
PD rating scale \cite{TLMR10}.

Another study described a weakly supervised multiple instance learning
approach to detecting symptoms of Parkinson’s Disease. This approach
addressed the issue of self-reporting resulting in inaccurate or
incomplete data \cite{DAlTH12}. Their algorithm learned to localize
symptoms to approximate, rather than exact, time ranges, making it
suitable for the sparse data that may result from incomplete reporting.

Gil and Johnson used a multilayer network with one hidden layer
and an output layer that output healthy or PD. The inputs were passed
through a sigmoidal activation function, and gradient descent
backpropagation was used to modify the weights. They achieved a
classification accuracy of 92.31\%.
They also trained an SVM using the sequential minimal
optimization (SMO) algorithm. SMO speeds up training
of SVMs, particularly those with non-linear kernel functions (Platt
1998), using a divide and conquer approach. Gil and Johnson used a
linear kernel with 91.79\% accuracy, and a Pearson VII function
kernel, with accuracy of 93.33\% \cite{GJ70}.

Mandal and Sairam also used a neural network with a sigmoidal activation function. They modified weights using backpropagation with dynamic
learning rate and momentum, and achieved an accuracy of 97.6471\%.
They also used SVM with a linear kernel and obtained an
accuracy of 97.6471\% \cite{MS12}.

% \chapter{Name of Chapter Two}
% \label{Chapter:Two}
% This is the first sentence of Chapter~\ref{Chapter:Two}.
% %\input litreview-shashmi.tex
% \chapter{Name of Chapter Three}
% \label{Chapter:Three}
% This is the first sentence of Chapter~\ref{Chapter:Three}.
% \chapter{Name of Chapter Four}
% \label{Chapter:Four}
% This is the first sentence of Chapter~\ref{Chapter:Four}.

\bibliography{srpaper-shashmi-biblio}


\end{document}
